\chapter{A parancsértelmező használata}%
\label{cha:shell}


A parancsértelmező a Unixos időszakból származik, azóta számtalan eltérő tudású
verziója jelent meg. Erre példa a \texttt{tcsh}, \texttt{ksh}, \texttt{bash},
\texttt{sh}. Ma ez utóbbin egy \textsc{posix} szabványban \cite{posix:shell}
rögzített tulajndoságokkal rendelkező változatot értjük, azonban előfordulhat,
hogy többet tud, például akkor, ha csupán egy symlink a \texttt{bash}-re.

A jegyzet hátralévő része a bash-ről szól. A bash a Bourne Again SHell
rövidítése.

A Linux rendszereken a bash az alapértelmezett parancsértelmező, ezért fontos,
hogy tisztában legyünk a működésével. E fejezet nagy vonalakban vázolja a
lehetőségeket, a későbbiekben pedig kitérünk az egyes részletekre,
kategóriánként csoportosítva. Mivel a bash önmagában csupán korlátozottan
használható, ezért kezdettől fogva fontos szerep jut a rendszer további,
bash-ből használható programjainak is.

\section{A bash alapjai}

Amikor karakteres felületen vagy éppen ssh-n keresztül jelentkezünk, akkor a
bash indul el, és várja, hogy parancsokat gépeljünk be számára. Bár a grafikus
felületen nem találkozunk vele, egy terminált megnyitva (amilyen az
\texttt{xterm}, vagy a \texttt{Konsole}) szintén e parancsértelmezőt indítjuk
el.

Ha már fut, akkor többféleképpen léphetünk ki belőle. az egyik lehetőség a
Ctrl-D leütése, amellyel jelezzük, hogy nem kívánunk további parancsokat kiadni,
vagyis a bemenete végetért. A másik az \texttt{exit} parancs -- utasítás --
begépelése, amely után enter-t ütve a bash befejezei futását. Ha nem indítottunk
el másik programot, mely lehet akár egy másik bash példány is, akkor a rendszer
kiléptet minket az ssh esetén, vagy pedig bezáródik a terminál ablaka.

A Ctrl-D a legtöbb program számára azt jelenti, hogy véget ért a bemenete,
vagyis a billentyűzetről nem fog újabb sorokat kapni, ám ezzel a bash esetén a
rendszerből is ki is lehet jelentkezni, ami lehet nem szándékos is, ezért a
\aref{cha:env-vars}.\ fejezetben látni fogjuk, hogyan lehet ezt elkerülni.

A futó parancsértelmező vár valamilyen bemenetre, amit értelmezni tud, majd
pedig lefuttatni, a parancs végét általában az enter billentyű leütésével
jelezhetjük -- ám van olyan parancs is, amely a későbbi sorokban folytatódik.
A hibásan beírt parancsot könnyen
javíthatjuk, mivel elegendő a kurzormozgató billentyűkkel balra-jobbra
navigálni, és beszúrni az új, már jó kódot, a régit pedig törölni. A korábban
kiadott parancsok is elérhetőek a fel-le gombokkal lépegetve közöttük. Sőt, az
eddig kiadott -- és megjegyzett -- parancsokat meg is tekinthetjük a
\texttt{history} parancs kiadásával. Egy korábban kiadott parancsot úgy is
megtalálhatunk, ha a C-r (control és r) billentyűk leütése után beírjuk a
parancs egy részletét.

\section{Parancsok általában}

A legtöbb parancs a következő formában működik:

\begin{VerbExample}
program opciók paraméterek
\end{VerbExample}

Ebből egyedül a \texttt{program} kötelező, a többi nem. Az opciók mínuszjellel
kezdődnek és egy karakter hosszúak, melyekangol ábécé kis- és nagybetűi, valamint
számjegyek lehetnek. Ez a régi Unixokon elterjedt verzió, ahol még fontos volt,
hogy minden minél rövidebben leírható -- begépelhető -- legyen. Ma már sokszor
az olvashatóság fontosabb, illetve egyes opcióknak nincs is egy karakteres
változata, ezek angol elnevezése ``long options'', hoyyszú opciók. Mindegyik két
darab mínuszjellel kezdődik, majd pedig legalább egy szó áll ott, több szó
esetén köztük mínuszjel áll. 

Erre példa a \texttt{wc} szűrő, amelynek az itt látható opciója a leghosszabb
beolvasott sor hosszát adja meg:

\begin{VerbExample}
wc -L
wc --max-line-length
\end{VerbExample} 

Egyes programok megengedik azt, hogy az opciók a paraméterek után legyenek.

Akár megengedik, akár nem, fontos, hogy meg lehessen őket különböztetni
egymástól, ezért a mínuszjellel kezdődő paraméter (például fájlnév) esetén e
határt valahogy jelölni kell.

Fájlnév esetén ez még egyszerű, hiszen a probléma megkerülhető az aktuális
könyvtár explicit használatával: ``./-file'', a gond akkor van, ha a paraméter
nem fájlnév. Erre egy speciális opció van, a mínuszjel: ``--''. Például az
összes fájl vagy könyvtár tulajdonságainak listázása -- lásd később:

\begin{VerbExample}
 ls -l -- *
\end{VerbExample}

Ez esetben az opció önállóan áll, ám a legtöbb opció összevonható, például:

\begin{VerbExample}
ls -l -A -R
ls -lAR
head -q -n 8
head -q -n8
head -qn8
\end{VerbExample} 

\subsection{Segítség a használható parancsokhoz}

Az összes programot nem lehet és nem is szükséges fejben tartanunk, ugyanis
számos lehetőséget nyújt egy Linux rendszer az információk megszerzésére. Ha
ismerjük a programot, akkor annak leírását, kézikönyvlapját a \texttt{man}
paranccsal kaphatjuk meg, vagy pedig az \texttt{info} paranccsal. Ha nem
ismerjük a parancs nevét, de tudjuk, mit csinál, akkor az \texttt{apropos}
paranccsal találhatjuk meg (ez egyébként ugyanaz, mint a \texttt{man
  -k}). Mindegyik esetben a keresendő program nevét paraméterben kell megadni,
esetleg a fejezetet is, ugyanis a kézikönyvlapok fejezetekbe tartoznak, és több
fejezetben is lehet azonos parancs.

A következőkben a kiadható parancsokra láthatunk példákat, mindegyiket egy-egy
rövid leírás követi a bash által értelmezett megjegyzés formájában, amely a
kereszt (hash mark, '\#') jeltől kezdve a sor végéig tart.

\begin{Verbatim}[xleftmargin=10mm]
man bash       # a bash kézikönyvlapja - nagyon hosszú
apropos copy   # mivel is tudunk másolni?
info coreutils # néhány alapvető parancs leírása
man 5 fstab    # a  /etc/fstab leírása az 5. fejezetben
\end{Verbatim}



\section{Ismerkedés a rendszerrel}

Az előző fejezetben már láthattuk, hogy hogyan néz ki a Linux fájlrendszere,
most megismerkedünk az ezen dolgozó parancsokkal, valamint néhány további
hasznos eszközzel.

\subsection{Fájlok, könyvtárak}

Az adataink egyetlen könyvtárban és annak alkönyvtáraiban találhatóak, ezt
nevezzük home könyvtárnak (home directory). Néhány rendszeren például a
honlapokat mégis egy másik helyen tároljuk, így valójában több helyen is
lehetnek a fájljaink, de nem tekinthető szokásosnak.

E könyvtárba a \texttt{cd} paranccsal léphetünk, egy másikba pedig hasonlóan,
csak meg is kell adnunk a könyvtár nevét:
\begin{VerbExample}[commandchars=\\\{\}]
cd \textit{könyvtárnév}
\end{VerbExample}
Azt, hogy hol állunk, a \texttt{pwd} paranccsal kérdezhetjük le, például:

\begin{VerbExample}
~ > pwd
/home/panther
\end{VerbExample}

A parancsot a \emph{prompt} után írhatjuk be, amelynek a végét általában egy
dollárjel (\$) szokta jelezni. A promptot néhol készenléti jelnek is nevezik. A
bash esetén szabadon módosítható a tartalma, ezért előfordulhat, hogy az
aktuális könyvtár nevét is tartalmazza, például Gentoo linuxok esetén a home
könyvtár nevét a `~' karakter (tilde, hullámvonal) helyettesíti:

\begin{VerbExample}
panther@tassadar ~/git/panther-elte/progkor/doc $ cd
panther@tassadar ~ $
\end{VerbExample}

Könyvtár létrehozására illetve törlésére az \texttt{mkdir} és \texttt{rmdir}
parancsok szolgálnak, mindegyik egy vagy több könyvtár nevét várja. Fájlok
törlésére pedig az \texttt{rm} parancs való:

\begin{VerbExample}
mkdir a a/b almafa  # ezen könyvtárak létrehozása
rmdir almafa        # egyből töröljük is
rm -r a             # az rm könyvtárat is töröl
\end{VerbExample}

A példákban az \texttt{rmdir} csupán üres alkönyvtárat tud törölni, ámde az
\texttt{a} könyvtár nem volt az, így kényelmesebb volt az \texttt{rm} parancsot
használni, ami az \texttt{a} tartalmát is törli a \texttt{-r} opció hatására. Ez
utóbbi jelentése rekurzív törlés: előbb a fájlokat, majd pedig az így üressé
vált könyvtárakat törli.

Fájlokat tipikusan egy \emph{editorral} (kisebb-nagyobb tudású
``szövegszerkesztő'' programmal) hozunk létre, vagy pedig másolással. Az
előbbiről \aref{sec:editors}.\ alfejezet foglalkozik. Az utóbbit a \texttt{cp}
paranccsal tehetjük meg, illetve ha áthelyezni, mozgatni szeretnénk, akkor pedig
az \texttt{mv} parancsot használhatjuk.

\begin{VerbExampleNum}
cp a b             # 1 fájl másolása
cp a b c/          # több fájl másolása
cp -r könyvtár  b  # könyvtár másolása
cp -ar könyvtár b  # ... jogok megőrzése
mv a b             # átnevezés/áthelyezés
mv -i a b          # interaktív
\end{VerbExampleNum}

E példákban az opcióknak nagy szerepük van. A \texttt{cp} önmagában csak fájlt
tud másolni, könyvtárat csupán a \texttt{-r} opcióval együtt a 3.\ példa
szerint. Ha több fájlt vagy könyvtárat másolunk, akkor az utolsó kötelezően
könyvtár, ebben lesz az új tartalom. A 4.\ példában a \texttt{-a} opció
garantálja, hogy a fájlok tulajdonságai sem változnak (tulajdonos, módosítási
idő és jogosultság). Az \texttt{mv} parancsnak nem szükséges megadnunk a
\texttt{-r} opciót -- nincs is ilyen --, ettől eltekintve ugyanaz igaz rá, mint
a \texttt{cp}-re. A 6.\ példában a \texttt{-i} opció hatására az \texttt{mv}
rákérdez minden egyes fájl esetén, ha a célfájl már létezik, vagyis a véletlen
felülírástól véd. Ez használható a \texttt{cp} és az \texttt{rm} parancsokkal
is, utóbbinál a véletlen törlés lehetőségét zárja ki.


\subsection{Környezet - más felhasználók}

A Unix rendszereken megszokott, hogy többen is használják egyszerre, tehát ha
egy szerverre ssh-n keresztül bejelentkezünk, akkor nagy valószínűséggel lesz
más felhasználó is. Próbaképpen nézzünk is szét:

\begin{VerbExample}
panther@panda:~$ users
dojo panther panther sz sz sz
panther@panda:~$ who
dojo     pts/1        Jan 27 09:53 (53d82d1e.adsl.enternet.hu)
sz       pts/3        Jan 27 11:47 (ia8vb9l7ul.adsl.datanet.hu)
sz       pts/4        Jan 27 11:49 (ia8vb9l7ul.adsl.datanet.hu)
panther  pts/5        Jan 27 11:51 (valerie.inf.elte.hu)
sz       pts/6        Jan 27 11:51 (ia8vb9l7ul.adsl.datanet.hu)
panther  pts/7        Jan 27 11:53 (dsl54008339.pool.t-online.hu)
panther@panda:~$ w
 11:53:25 up 9 days, 19:54,  6 users,  load average: 0.02, 0.01, 0.00
USER     TTY      FROM              LOGIN@   IDLE   JCPU   PCPU WHAT
panther@panda:~$ uptime
 11:53:28 up 9 days, 19:54,  6 users,  load average: 0.02, 0.01, 0.00
\end{VerbExample}

Ezen parancsok sokféle információt tartalmaznak, tekintsük sorban. A legelső a
\texttt{users} volt, amely a bejelentkezett felhasználók neveit írja ki. Ha
valakinek az azonosítója egynél többször szerepel, akkor ugyanennyiszer
jelentkezett be. A következő a who, amely a felhasználók azonosítóját, az
általuk használt terminált (a \texttt{/dev} alatti fájl nevét), a bejelentkezés
idejét, valamint azon gép nevét listázza, ahonnan bejelentkezett. A \texttt{w}
első sora az \texttt{uptime} parancs kimenetét tartalmazza, a többi pedig
további adatokat közöl a bejelentkezett felhasználókról. Ha grsecurity
\cite{grsecurity}, valamint pax \cite{pax} kiegészítésekkel biztonságosabbá tett
rendszeren adjuk ki, akkor gyakran üres a lista. Egy otthoni gépen viszont már
többet láthatunk:

\begin{VerbExample}
panther@tassadar ~ $ users
panther panther
panther@tassadar ~ $ who
panther  :0           2008-01-27 09:52
panther  pts/7        2008-01-27 11:59 (artanis.panthernet)
panther@tassadar ~ $ w
 11:59:09 up  2:08,  2 users,  load average: 0,16, 0,14, 0,13
USER     TTY        LOGIN@   IDLE   JCPU   PCPU WHAT
panther  :0        09:52   ?xdm?   5:43   0.00s /bin/sh /usr/kde/3.5/bin/startkde
panther  pts/7     11:59    7.00s  0.16s  0.16s -bash
panther@tassadar ~ $ uptime
 11:59:12 up  2:08,  2 users,  load average: 0.16, 0.14, 0.13
\end{VerbExample}

A \texttt{:0} azt jelzi, hogy az adott felhasználó grafikus felületen
jelentkezett be, a nullás azonosítójú képernyőn -- több grafikus felület is
futhatna egyszerre. A ``LOGIN@'' a bejelentkezési időt adja meg, az ``IDLE'' azt
az időt, amennyi ideje inaktív. Grafikus felület esetén ilyen nincsen, ám ssh
esetén már van, ez látszik is a példában, 7 másodperce inaktív a felhasználó --
csak a példa kedvéért bejelentkeztem egy másik gépről. A JCPU és a PCPU a
felhasznált processzoridőt adja meg, a WHAT pedig a futtatott
programot. Természetesen nincsen olyan, hogy \texttt{-bash}, a mínusz jelnek
jelen esetben speciális jelentése van, méghozzá az, hogy interkatív vagyis
bejelentkezési shellről van szó, ahova a felhasználó gépelhet.

Az \texttt{uptime} kimenete sokszor érdekesebb, az első része a pontos időt írja
ki, majd az ``up'' szó után a futási időt, például 2 óra 8 perce megy a gép,
vagy éppen a panda esetén 9 napja, 19 órája és 54 perce, továbbá 2 illetve 6
bejelentkezett felhasználó van. A ``load average'' az átlagos terhelés az elmúlt
egy, öt, valamint 15 percben, vagyis átlagosan hány folyamat (futó program)
kívánta használni a processzorokat. Amíg ezek az értékek legfeljebb akkorák,
mint ahány processzor -- pontosabban processzormag -- van a gépben, addig
nincsen gond, utána viszont már túlterhelt. Ez előfordulhat akkor, ha valamelyik
program hibásan működik, vagy túl sokan használják a gépet, de lehet
természetesnek nevezhető állapot is, amikor átmenetileg nagy számításigényű
feladatot hajt végre valamely program.

Az egyes felhasználókról további információkat is megtudhatunk a finger parancs
kiadásával:

\begin{VerbExample}
panther@tassadar ~ $ finger
Login    Name                   Tty      Idle  Login Time   Office     Office Phone
panther  Tóth László Attila *:0             Jan 27 09:52
panther  Tóth László Attila  pts/7      13  Jan 27 11:59 (artanis.panthernet)
panther@tassadar ~ $ finger -m panther
Login: panther                          Name: Tóth László Attila
Directory: /home/panther                Shell: /bin/bash
On since Sun Jan 27 09:52 (CET) on :0 (messages off)
On since Sun Jan 27 11:59 (CET) on pts/7 from artanis.panthernet
   14 minutes 12 seconds idle
New mail received Wed Jan 23 21:52 2008 (CET)
     Unread since Sat Jan  5 15:30 2008 (CET)
No Plan.
\end{VerbExample}

Paraméterek nélkül kiadva a bejelentkezettt felhasználók nevét is kiírja. A
következő példában a \texttt{-m} opció nélkül nem csak a felhasználónevet
keresné, hanem a valódi névben is megnézné, hogy az adott karakterosozat
szerepel-e, így további felhasználókat is listázna, így viszont a pontos
illeszkedést kapjuk. A listázott adatok: azonosító (login), név, home könyvtár,
parancsértelmező (shell), mikor, melyik terminálon és honnan jelentkezett be. A
``messages off'' azt jelenti, hogy nem fogad el üzeneteket, ugyanezt jelenti az
első példában a ``TTY'' oszlopban látható csillag karakter is.

Kiírja, hogy van-e levele, és a \texttt{.plan} fájl tartalmát is, feltéve, ha el
tudja olvasni. Nálam nincsen ilyen fájl, ezért a ``No plan'' szöveg jelenik meg.

Az üzenetek tiltását és engedélyezését a \texttt{mesg} paranccsal lehet
szabályozni, a \texttt{y} paraméterrel engedélyezni, \texttt{n}-nel pedig
tiltani:

\begin{VerbExample}
mesg y              # engedélyezés
mesg n              # tiltás
\end{VerbExample}


Hogyan tudunk egy másik felhasználónak írni? A Unix világban már nagyon
hosszúnak számító \texttt{write} paranccsal. Hosszú, ugyanis általában két,
esetleg három karakteres a legtöbb parancs. Paramétere annak a felhasználónak az
azonosítója, akinek gépelni szeretnénk. Mivel a program a terminálra ír,
miközben a másik rajta dolgozik, ezért úgy illik, hogy valahogyan jelezzük,
mikor fejeztük be azt, amit épp írunk, és ekkor a másik elkezdhet írni. Enélkül
a másik félig elkezdett sorában fog megjelenni az üzenetünk, ami nagyon zavaró.
\ut{write valaki}
A programból a Ctrl-D leütésével léphetünk ki.

\subsection{Környezet - a számítógép}

Fentebb láthattuk, hogyan lehet megtudni, mennyi ideje megy a számítógép, és
mekkora a terhelése. Erre szolgált az \texttt{uptime} parancs. Azonban további
információkat is meg lehet tudni, így a \texttt{df} az egyes partíciók
összméretét, foglalt és szabad részének méretét írja ki:

\begin{VerbExample}
panther@tassadar ~ $ df
Fájlrendszer          1K-blokk   Foglalt    Szabad Fo.% Csatl. pont
/dev/sda3            194396620 167506840  26889780  87% /
udev                     10240       224     10016   3% /dev
shm                    2023320         0   2023320   0% /dev/shm
panther@tassadar ~ $ df  -h
Fájlrendszer         Méret  Fogl. Szab. Fo.% Csatl. pont
/dev/sda3             186G  160G   26G  87% /
udev                   10M  224K  9,8M   3% /dev
shm                   2,0G     0  2,0G   0% /dev/shm
panther@tassadar ~ $ df  --si
Fájlrendszer          Méret  Fogl.  Szab.  Fo.% Csatl. pont
/dev/sda3              200G   172G    28G  87% /
udev                    11M   230k    11M   3% /dev
shm                    2,1G      0   2,1G   0% /dev/shm
panther@tassadar ~ $ df  -m
Fájlrendszer          1M-blokk   Foglalt    Szabad Fo.% Csatl. pont
/dev/sda3               189841    163581     26260  87% /
udev                        10         1        10   3% /dev
shm                       1976         0      1976   0% /dev/shm
\end{VerbExample}

Általában ezekkel az opciókkal érdemes használni, alapesetben 1024 bájtos
blokkal számol. A \texttt{-h} hatására jól olvasható módon írja ki, továbbra is
1024-es szorzóval (K = kiB, M = MiB, G=GiB stb.). A \texttt{--si} ugyanez, ezres
szorzóval (vagyis itt a K kB-ot jelent stb.), a \texttt{-m} pedig MiB-ban adja
meg a méretet.

A rendelkezésre álló memóriát is megkaphatjuk a \texttt{free} segítségével,
opciók nélkül 1 kiB-os blokkok számában, \texttt{-m} esetén pedig MiB-ban.

\begin{VerbExample}
panther@panda:~$ free
             total       used       free     shared    buffers     cached
Mem:        246176     195508      50668          0          0      83092
-/+ buffers/cache:     112416     133760
Swap:       506008       3016     502992
panther@panda:~$ free -m
             total       used       free     shared    buffers     cached
Mem:           240        190         49          0          0         81
-/+ buffers/cache:        109        130
Swap:          494          2        491
\end{VerbExample}

Végezetül magának a hardvernek az adatait is megtekinthetjük a
\texttt{/proc/cpuinfo} fájl listázásával és az \texttt{lspci} használatával. A
példában szereplő lista nem teljes és a sorok végét sem tartalmazza.

\begin{VerbExample}
panther@panda:~$ cat /proc/cpuinfo
processor       : 0
vendor_id       : GenuineIntel
cpu family      : 6
model           : 8
model name      : Celeron (Coppermine)
stepping        : 10
cpu MHz         : 906.741
cache size      : 128 KB
fdiv_bug        : no
hlt_bug         : no
f00f_bug        : no
coma_bug        : no
fpu             : yes
fpu_exception   : yes
cpuid level     : 2
wp              : yes
flags           : fpu vme de pse tsc msr pae mce cx8
bogomips        : 1816.05

panther@panda:~$ lspci
00:00.0 Host bridge: VIA Technologies, Inc. VT82C693A/694x [Apollo PRO 
00:01.0 PCI bridge: VIA Technologies, Inc. VT82C598/694x [Apollo MVP3/
00:07.0 ISA bridge: VIA Technologies, Inc. VT82C686 [Apollo Super South]
\end{VerbExample}

\section{Editorok}%
\label{sec:editors}

Amikor egy-egy szövegfájlt, szkriptet vagy éppen valamely program egyik
forrásfájlját módosítjuk, akkor hasznos, ha a használt
\emph{editor} elég nagy tudású ahhoz, hogy támogassa munkánkat. Egy rendszeren
ebből sokféle van, a legminimálisabb tudásútól (\emph{mcedit}, \emph{pico})
a legtöbbet tudóig (\emph{ViM}, \emph{Emacs}). Az előbbit illik ismerni, hiszen
lényegében bármely Unix változat elé ülünk le, vagy ez, vagy pedig ennek az őse,
a \emph{vi} biztosan fenn lesz az adott gépen.

\subsection{A vim}

A vim alapvetően két módban futhat, az egyikben parancsokkal lehet vezérelni, a
másikban pedig a szöveget módosítani. A váltás közöttük például az \texttt{ins}
és az \texttt{esc} billentyűk használatával lehetséges.

A ``parancsmód'' főbb lehetőségeit \aref{tab:vim-ex}.\ táblázat foglalja össze.

\begin{table}\center%
\begin{tabular}{ll}\toprule
Billentyű & Jelentés\\\midrule
:q & kilépés\\
:q!& kilépés mentés nélkül\\
:w & mentés\\
:syntax on & syntax highlight -- kódszínezés\\
:!parancs & parancs futtatása a shellben\\
v & kijelölés\\
shift-v & sorok kijelölése\\
\bottomrule
\end{tabular}
  \caption{A vim néhány billentyűkombinációja}
  \label{tab:vim-ex}
\end{table}


\subsection{Az emacs}

Az Emacs (helyesen kiejtve kb.\ imeksz) egy több évtizedes múlttal rendelkező
editor, hiszen lisp-ben íródott. Magát a programot kevesen fejlesztik, ám
\emph{lisp} nyelven bárki által programozható, és számtalan kiegészítés létezik
hozzá. A vi-hoz hasonlóan szokatlan a működése, mert bár nincs megkülönböztetve,
hogy mikor lehet parancsot kiadni, s mikor lehet szöveget szerkeszteni, a
vezérlése billentyű-kombináció sorozat révén lehetséges. Ezek megértéséhez
tudnunk kell, hogy a Control gombot C-vel, a Meta gombot pedig M-mel
rövidíti. Ez utóbbi \textsc{pc}-k esetén az Alt gomb, míg Mac esetén a
Command. Például C-x C-s a mentés.

\begin{table}\center
\begin{tabular}{ll}\bottomrule
Billentyű-kombináció &Jelentés\\\midrule
C-x C-c & kilépés\\
C-x C-s & mentés\\
C-x C-f & másik fájl megnyitása a név megadásával\\
C-x C-w & mentés másik név megadásával\\
C-SPACE & kijelölés kezdete\\
C-x w   & kivágás\\
C-x y   & beillesztés\\\midrule
C-x b   & ugrás másik fájlra (pufferre)\\
C-x 1   & 1 részre osztott ablak\\
C-x 2   & vízszintesen tagolt ablak\\
C-x 3   & függőlegesen tagolt ablak\\
\bottomrule
\end{tabular}
  \caption{Emacs vezérlése}
  \label{table:emacs}
\end{table}



\section{Néhány példa}

A bash működése viszonylag egyszerű, ám összetett, bonyolult programok --
szkriptek -- is viszonylag könnyedén írhatóak segítségével is, melyek elsősorban
szövegfeldolgozásra, vagy pedig gyakran ismétlődő tevékenység elvégzésére
szolgálnak.

A következő példa azt mutatja meg, hogy egyszerű eszközöket felhasználva egy
bonyolultabb erdményt is megkaphatunk:

\begin{VerbExampleNum}
panther@panda:~$ cat /proc/cpuinfo | grep cpu\ MHz
cpu MHz         : 906.741
panther@panda:~$ grep cpu\ MHz /proc/cpuinfo | cut -f2 -d:
 906.741
\end{VerbExampleNum}

Az első parancsban a \texttt{cat} immár ismerős, ennek kimenetét módosítja a
\texttt{grep}, a mögötte álló szöveget ('cpu MHz') tartalmazó sorokat hagyja
csak meg, azaz megszűri. Innen kapták a \texttt{grep}-hez hasonló programok a
\emph{szűrő} elnevezést, hiszen a bemenetüket megszűrik vagy
átalakítják. Majdnem mindegyik elfogad fájlneveket is, ezért a \texttt{cat}
paranccsal nem is szükséges kiiratnunk a tartalmukat, hogy e kimenetet
használjuk fel egy másik program bemeneteként. Ez utóbbi esetben a két parancs
egy \emph{csővezezetéket} -- angolul: pipe -- alkot, és a köztük álló
karakternek is ez a neve (és magyar billentyűzet-kiosztás esetén egy
\textsc{pc}-n az AltGr-W ``billentyű-kombináció'' csalja elő). A \texttt{cut} a
második mezőt hagyja csak meg ebben a példában, a mezőelválasztó karakter pedig
a kettőspont volt.

Az összes ``.txt'' végződésű fájl nevének kiírássára is számos lehetőség van,
például:

\begin{VerbExampleNum}
~ $ ls *.txt
a.txt b.txt
~ $ ls -1 *.txt
a.txt
b.txt
~ $ find . -name '*.txt'
./a.txt
./b.txt
~ $ for i in *.txt; do echo $i; done
a.txt
b.txt
~ $ find . -iname '*.txt'
a.txt
a.TXT
b.txt
\end{VerbExampleNum}

Az első két parancs esetén a ``*.txt'' önmagában, idézőjel vagy aposztróf nélkül
áll, ekkor a bash behelyettesíti az ott álló fájlneveket, tehát az első parancs
azonos az

\begin{VerbExample}
ls a.txt b.txt
\end{VerbExample}

\noindent kiadásával, amikor listázzuk e fájlokat. További tulajdonságokat az
\texttt{ls -l} paranccsal lehet megtekinteni, az \texttt{-l} opció esetén a
jogosultságokat, tulajdonost, csoportot és módosítási időt is láthatjuk. A
második példában a \texttt{-1} opció hatására minden fájlnév külön sorba kerül.

Az ötödik sorban a \texttt{find} az aktuális könyvtárban -- melynek jele az egy
pont -- és annak alkönyvtáraiban keresi az összes \texttt{.txt} végződésű fájlt
és könyvtárat. Ekkor a ``*.txt'' aposztrófok között áll, ezért a program
változtatás nélkül, ebben a formában kapja meg. A 12.\ sorban lévő változat a
\texttt{-iname} paramétert használja, ekkor a fájlnévben lévő kis-és nagybetűk
közt nem tesz különbséget, bár a Linux alapvetően megkülönbözteti a kis- és
nagybetűket.

A 9.\ sorban lévő for utasítás (ciklus) ugyanazt írja ki, mint az \texttt{ls
  -1}, csak bonyolultabb az alakja, amely más feladatoknál lényeges. Az
\texttt{i} egy változónév, amely ekkortól létezik, és a \texttt{\$i} formában
kapjuk meg az értékét. Az \texttt{in} után egy lista szerepel, amely minden
elemének értékét egyszer felveszi az \texttt{i} változó. Mint fentebb láthattuk,
a bash itt is kifejti a \texttt{*.txt} kifejezést, vagyis az \texttt{in a.txt
  b.txt} az, ami ténylegesen le is fut. A cklus magja a \texttt{do} és
\texttt{done} kulcsszavak között szerepel.

A fenti parancsok között van még egy lényeges különbség: a \texttt{find} esetén
aposztrófok között szereplő ``*.txt'' biztosan működik, míg nagyon sok fájl
esetén az összes többi nem, ugyanis ekkor nem fér el a parancssorban, egy
utasítás hossza fájlnévvel, mindennel együtt nagyjából 16 KiB méretű lehet.

\section{Szkriptek}
\label{sec:scripts}

Egy szkriptfájl egy olyan szöveges állomány, amely valamely program által
értelmezhető utasításokat -- parancsokat -- tartalmaz. Ahhoz, hogy ezt
programként használhassuk, futtathatóvá kell tenni. Amely program képes egy
parancsokat tartalmazó fájlt feldolgozni, ahhoz írható szkriptfájl
(parancsfájl). 

Valójában a kettő nem teljesen ugyanaz, mivel a parancsokat tartalmazó állományt
paraméterként megadjuk az adott programnak, míg a szkriptfájlt le is tudjuk
futtatni, anélkül, hogy lényeges lenne, mely program dolgozza fel. A bash is
ilyen program, valamint a sed is, -- ez utóbbi alapvetően szöveghelyettesítésre
szolgál és róla bővebben \aref{cha:szűrők}.\ fejezetben lesz szó, -- amely a
\texttt{-f} opció után vár egy fájlnevet. 

Mi a bash-t használjuk parancsértelmezőnek, így felmerül a kérdés, hogy a bash
honnan tudja, melyik programhoz tartozik a szkript?  A legelső sor határozza
meg, hogy melyik parancs fogja lefuttatni, melynek a szerkezete:
\texttt{\#!programnév}, vagy ha további opciók szükségesek, akkor
\texttt{\#!programnév opciók}. Ez a bash esetén \texttt{\#!/bin/bash}, a sed
esetén pedig \texttt{\#!/bin/sed -f}. Ha a program neve nem tartalmaz teljes
elérési utat, akkor a PATH környezeti változóban megadott könyvtárakban keresi.

A bash rávehető arra, hogy kiírja mit is csinál a \texttt{-x} opció
használatával, mely hibakareséshez hasznos lehet, ha valamiért meglepő az
eredmény, vagy egyszerűen csak nem a várt módon viselkedik a parancsfájl.

Mint minden programnak, a szkripteknek is lehetnek paraméterei, ezekre a
\texttt{0}, \texttt{1} stb.\ nevű változókra lehet hivatkozni. A \texttt{\$0} az
az érték -- egészen pontosan programnév -- amelyikkel a szkriptünket
meghívták. Az első paramétert a
\texttt{\$1} megadásával használhatjuk és így tovább a kilencedik paraméterig. A
tizediktől kezdve már picit más jelölést alkalmazunk (részletesebben lásd
\aref{sec:vars-eval}.\ fejezetben), ugyanis itt a \texttt{\$10} és
\texttt{\$\{10\}} között nagy különbség van. Az előbbi az első paraméter értéke
és utána egy '0' karakter, az utóbbi pedig a 10.\ paraméter értéke.


% Local Variables:
% fill-column: 80
% mode: latex
% End:
