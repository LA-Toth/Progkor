\chaptern{Bevezetés}
\label{cha:bevezetes}

A \emph{Programozási környezet} tárgy keretein belül a \textsc{html} alapjait,
\textsc{unix} shell szkripteket (konkrétan Linuxokon Bash-t) valamint
\textsc{vms}-t oktatunk. E jegyzet az előbbi kettőről szól, a \textsc{html}
alapjairól és minimális \textsc{css}-ről, valamint részletesen a Bash szkriptek
készítéséről.

A jegyzet felépítése a következő: a \textsc{html}-ről, annak is az
\textsc{xhtml} változatáról egyetlen fejezetben a -- \aref{cha:html}.\
fejezetben --, a Linuxról és a parancssorról pedig a többi fejezetben. A Linux
és a Bash alapjairól szól \aref{cha:linux}.\ és \aref{cha:shell}.,
 a szövegfeldogolzásban
használható programokról, a szűrőkről \aref{cha:szűrők}., a
környezeti változókról -- használatukról és a lényegesebb előre definiáltak
jelentéséről \aref{cha:env-vars}., a vezérlési szerkezetekről -- elágazás,
szekvencia és ciklus -- \aref{cha:vez-szerk}., a reguláris kifejezésekről
\aref{cha:regex}., végezetül az awk-ról, és egyéb, az előző fejezetekben el nem
hangzott lehetőségekről \aref{cha:others}.\ fejezetben.

Néhány fejezet elején szerepel történeti áttekintés, amely csupán tájékoztató
jellegű és nem mindig pontos.


% Local Variables:
% fill-column: 80
% mode: latex
% End:
