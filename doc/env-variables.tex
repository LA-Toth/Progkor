\chapter{Környezeti változók}%
\label{cha:env-vars}

Bár a parancsokat használhatjuk fix paraméterekkel is, gyakori, hogy ismételten
kell ugyanazt a programot minimális eltéréssel futtatni. Tipikusan a
paraméterek, munkakönyvtár és egyes fájlnevek lehetnek eltérőek. Ezen
eltéréseket könnyen kezelhetjük a környezeti változókkal, amelyeknek van neve és
értéke, így elég csak egyszer megfelelően beállítani őket, később a név
ismeretében megírhatjuk a már általánosan használható kódrészletet.

Leginkább szkriptek esetén találkozhatunk velük, hiszen azok a paramétereiket
környezeti változók formájában érhetik el. Vagy csupán e paraméterek
befolyásolják a viselkedést, vagy pedig további változók, amelyeket tipikusan a
szkript elején állíthatunk be -- például ha egy másik rendszeren a szükséges
állományok és programok máshol találhatóak.

A grep, sed és még számos egyéb eszköz a /bin könyvtárban található, azonban
előfordulhat, hogy az adott változat már túl régi, ezért fordítunk magunknak egy
újat, ekor már meg kell adnunk azt is, hogy melyik programra gondoltunk:

\begin{VerbExample}
#!/usr/bin/env bash

GREP=/bin/grep
CUT=/bin/cut
CFGFILE=/etc/pelda.cfg

host=$($GREP host $CFGFILE | $CUT -f1 -d= )
\end{VerbExample}

\indent és ekkor a \texttt{host} valamely számítógép nevét tartalmazza. Azonban
ez sokszor nem éppen elegáns megoldás, ezért inkább azt mondjuk, hogy a
programokat először ne a \texttt{/bin} könyvtárban keresse, hanem például a
/usr/local/bin-ben. Ezt a \texttt{PATH} környezeti változó beállításával
tehetjük meg:

\begin{VerbExample}
 #!/usr/bin/env bash

PATH=/usr/local/bin:/bin:/usr/bin
CFGFILE=/etc=pelda.cfg

host=$(grep host $CFGFILE | cut -f1 -d= )
\end{VerbExample}

A továbbiakban e változók működéséről lesz szó, arról, hogy hogyan állítsuk be
őket, milyen értéket vehetnek fel, és végül, hogy hogyan kaphatjuk meg az
értéküket, vagy csupán annak egy részét.

\section{A környezeti változók szintaxisa valamint beállítása}
A változók neve a legfontosabb tulajdonságuk, mely nem állhat bármilyen
karakterból, hanem csupán az angol ábécé kis- és nagybetűiből, számjegyekből
valamint az alulvonás karakterből (A-Z, a-z, 0-9 és \_). A beállításkor ügyelni
kell arra, hogy az értéke ne tartalmazzon szóközt, vagy ha mégis, akkor
idézőjelek vagy aposztrófok között kell megadnunk az értéket, mely ezáltal védve
van a shelltől.

\begin{VerbExampleNum}
NEV=érték
NEV="hosszabb érték $NEV"
NEV='hosszabb érték $NEV'
NEV=
unset NEV
export NEV
export NEV=érték PATH
\end{VerbExampleNum}

Az első négy művelet egy \texttt{NEV} nevezetű változónak állít be értéket,
az ötödik megszünteti a változót, az utolsó kettő pedig exportálja.

Most tekintsük meg, hogy pontosan mi történik az egyes parancsok kiadása
után. Az első beállít egy szóközt nem tartalmazó értéket, ezért nem szükséges
aposztróffal vagy idézőjellel védeni az értéket. Viszont csak akkor működik az
értékadás (mindegyik esetben), ha a változó neve és érteke között pontosan egy
egyenlőségjel áll, azon kívül pedig semmi. A második ugyanezt teszi, csak
idézőjel védi az (esetleges) szóközöket, mindkét esetben az értékben szereplő
változóbehelyettesítések megtörténnek, azaz ha egy dollárjel (\$) után egy
változó neve áll, akkor annak értéke kerül a helyére. A \texttt{NEV} változó
értéke ezzel "\texttt{hosszabb érték érték}" lesz, hiszen előtte az
"\texttt{érték}" volt a tartalma.

Ettől eltér a harmadik eset működése, mert ott nem idézőjel, hanem aposztróf
van, amelynek az a jelentése, hogy pontosan a megadott érték kerüljön a
változóba, jelen eseteben ez a "\texttt{hosszabb érték \$NEV}" szöveg, a
változók
értéke nem helyettesítődik be. A helyzet tovább bonyolítható, hiszen írható
ilyen is (átmenetileg megszakítva az aposztróffal vagy idézőjellel védett
értéket): \texttt{NEV='valamilyen \$szoveg'\$VALTOZO"a b c \$V2"}, ahol az
idézőjeleken belüli, és az azt megelőző változók (\texttt{V2} és
\texttt{VALTOZO}) értéke behelyettesítődik, az aposztrófok közti pedig nem
(\$szoveg esetén a \$ jel megmarad).



\section{Néhány fontosabb környezeti változó}

A mindennapi munkánk során számos változót használunk, és
jónéhány előre definiált változóval találkozhatunk, és mi is létrehozhatunk
továbbiakat. Készen kapjuk a home könyvtár helyét megadó változót, \texttt{HOME}
néven, amelyet a bejelentkezési azonosító (a példában \texttt{user}) alapján nem
biztos, hogy kitalálhatunk. Általában \texttt{/home/user} az értéke, a pandán
viszont ez \texttt{/h/u/user}. Megint csak a legtöbb esetben minden adatunk e
könyvtárban található, bizonyos esetekben azonban célszerűbb külön könyvtárba
tenni tipikusan a weblapokat és egyéb publikus fájlokat, hiszen ekkor már
beállítható, hogy a home könyvtárat egyedül a tulajdonos érhesse el. A
weblapokat alapesetben a \texttt{\$HOME/public\_html} könyvtár tartalmazza, a
pandán viszont máshol van, lásd a következő bekezdést.

Egy másik a bejelentkezési azonosítónk, a \texttt{USER} változóban, így már a
\texttt{HOME} könyvtárat másképp is megkaphatjuk: \verb|/home/$USER|, illetve
a pandorán \verb|/h/${USER:0:1}/$USER|, ugyancsak itt a honlap helye a
\verb|/h/public/${USER:0:1}/$USER/public_html|.

Az aktuális könyvtárat a \texttt{PWD} változó tartalmazza, amely megkapható a
\texttt{pwd} parancs kiadásával is. Az előző munkakönyvárat az \texttt{OLDPWD}
tartalmazza, azonban e változó nélkül is visszaléphetünk a korábbi könyvtárba:
\begin{VerbExample}
cd -
cd ~-
\end{VerbExample}

A fájlrendszer szerkezetét láthattuk a \aref{cha:linux}.\ fejezetben, így fixen
megadhatnánk az általunk használt programok nevét is, például
\texttt{/bin/grep}, ám ha nem pont itt található (az általunk igényelt
változat), akkor nem fog a szkriptünk működni. Ezért elég csak a program nevét
(például grep) ismernünk, a pontos helyét pedig egy újabb, korábban definiált
változóból tudhatjuk meg, amelynek \texttt{PATH} a neve. Kettősponttal
elválasztott könyvtárnév-sorozatot tartalmaz, minimálisan a következőt:
\texttt{/bin:/usr/bin}. A parancsértelmező használja ezt, amikor egy futtatandó
programnak csupán a nevével találkozik, ilyenkor megnézi a legleső könyvtárat,
hogy benne van-e. A \texttt{grep} épp igen, ezért le is futtatja. Ha nem ott
lenne, akkor a következő könyvtárban, a \texttt{/usr/bin}-ben keresné.

Ugyanezen elven működik a \texttt{CDPATH} változó is, amelyet ritkábban
használunk, ám olykor nagyon hasznos. Tipikus értéke: \texttt{.:\~{}:Documents},
vagyis a \texttt{cd} először az aktuális (más néven munka-) könyvtárban keres,
majd a home-ban (ennek a rövidítése a ``\~{}'', tilde karakter), végül egyéb
felsorolt könyvtárakban. Természetesen ez is csupán akkor működik, ha a
könyvtárnév nem abszolút, azaz nem perjellel kezdődik, viszont a paraméterben
lehet perjel is.

A \texttt{PATH} változó azonban nem csak a parancsokhoz használható, hanem a
szkriptek esetén is, ugyanis  a \texttt{.}\ vagy a \texttt{source} parancs is
ezt a változót használja, azonban csakis abban az esetben, ha az aktuális
könyvtárban nem szerepel az adott fájl:

\begin{VerbExample}
. almafa    
\end{VerbExample}

\noindent jelentése: az aktuális parancsértelmezőben fusson le az
\texttt{almafa} nevű
fájlban szereplő összes parancs. Először a munkakönyvtárban keresi ezt a fájlt,
majd ha itt nincsen, akkor a \texttt{PATH}-ban felsorolt könyvtárakat nézi
végig, hátha valamelyikben szerepel az \texttt{almafa}.

Az \texttt{UID} változó tartalmazza a felhasználónevünkhöz rendelt számot, a
\texttt{GROUPS} pedig a csoportjainkhoz. Minden felhasználó legalább egy csoport
tagja, de tartozhat többe is, ezáltal csoportszinten lehet szabályozni, hogy ki
milyen fájlokhoz és könyvtárakhoz fér hozzá.

Az a szöveg, ami után a parancsokat begépelhetjük, a \emph{prompt}, ennek értéke
a \texttt{PS1} változóban definiált, erre később még visszatérünk.

A programok és szkriptek futtatásához kötődik a  \texttt{\$}, \texttt{?} és a
\texttt{!} változó. Az első az aktuális folyamat (processz, mely most a
parancsértelmező) azonosítója. Ez mindig egyedi az adott rendszeren, és
indulásonként más és más. Az értékét például így írhatjuk ki (vagyis a dollárjel
a változó \emph{neve}):

\begin{VerbExample}
echo $$
\end{VerbExample}

\noindent A kérdőjel változó az utoljára lefutott folyamat \texttt{státuszát}
tartalmazza, amely 0 sikeres befejeződés esetén, bármilyen  hiba előfordulásakor
viszont nemnulla. A felkiáltójel változó pedig az utoljára háttérben elindított
program processz azonosítója (\textsc{pid}-je).

Végül tekintsünk még két további változót:
a \texttt{COLUMNS} és a \texttt{LINES} változó tartalmazza, hogy a terminálunk,
amelyen éppen fut a parancsértelmező, hány oszlopot és hány sort képes
megjeleníteni egyszerre. Alapesetben ez a méret 80x25, és a programkódoknál
gyakran követelmény, hogy ekkor is látszódjék a teljes sor.


\subsection{Prompt}

A prompt, amely után a parancsokat beírjuk, a \texttt{PS1} változóban állítható
be. Néha több sorba fér csak ki, amit írunk, ekkor a \texttt{PS2}-ben beállított
szöveg van minden sorban (tipikusan \texttt{'> '}). Az utolsó ilyen változó a
\texttt{PS4}, amely akkor jelenik meg, amikor a bash-t a \texttt{-x} opcióval
indítottuk, \texttt{'+ '} a szokásos értéke.

Mind a háromféle prompt ugyanolyan módon állítható, néhány speciális jelentéssel
bíró karaktersorozat is van, például az azonosító, a gép neve, vagy éppen az
aktuális idő, ezekre néhány példát \aref{table:prompt}.\ táblázat tartalmaz,
további részletek a \textit{bash(1)} kézikönyvlapban.

\begin{table}\centering
\begin{tabular}{@{}cl@{}}\toprule
Speciális karakter & Jelentés\\\midrule
\verb|\u| & azonosító (\$USER)\\
\verb|\h| & gép neve (\$HOSTNAME), az első '.'-ig\\
\verb|\s| & a shell neve (basename \$0)\\
\verb|\t| & pontos idő, ÓÓ:PP:MM formában\\
\verb|\w| & az aktuális könyvtár neve (basename \$PWD)\\
\verb|\W| & az aktuális könyvtár\\
\verb|\\| & a '\textbackslash' karakter\\
\verb|\#| & hanyadik kiadott parancs lesz\\\bottomrule
\end{tabular}
\caption{Példák a prompt lehetséges elemeire}
\label{table:prompt}
\end{table}



\section{A környezeti változók többféle behelyettesítése}
\label{sec:vars-eval}

Az eddigiek alapján \texttt{\$NEV} beírásával a \texttt{NEV} környezeti változó
értékét használjuk fel, ez azonban csupán a legegyszerűbb formája. További
lehetőségek is vannak, az egyik, hogy jelöljük, mettől meddig tart a változó
neve, a többi pedig ennek kibővített alakja. Egy változó egy részét is
felhasználhatjuk, megadhatunk alapértelmezett értéket, ha nincs beállítva a
változó, egyes részeit törölve az új értékkel dolgozhatunk, és hasonlók.

A változó nevét kapcsoszárójelek közé téve egyértelműen jelezzük, hogy mettől
meddig tart a név: \texttt{\$\{NEV\}}. Ez lényeges, hiszen mást jelent a
\texttt{\$\{NEV\}\_1} és a \texttt{\$NEV\_1}: az előbbi a \texttt{NEV} változó
értéke, majd pedig az \texttt{\_1} karakterek, az utóbbi viszont egy értelmes
változónév, így a bash a lehető leghosszabb elnevezést keresi, amely jelen
esetben a \texttt{NEV\_1}. Tehát a bash ezen nevű változó értékét fogja
behelyettesíteni.

Lehetőség van az érték egy szeleetének felhasználásra, például 
a második karaktertől a végéig: \texttt{\$\{NEV:1\}},
harmadiktól: \texttt{\$\{NEV:2\}} stb. Lényeges, hogy a legelső karakter indexe
a nulla (0). Megadható, hogy hány karaktert szeretnénk kiírni, pédául a
nulladikat (azaz a legelsőt) és a 3-5.\ karaktereket:

\begin{VerbExample}
echo ${NEV:0:1} ${NEV:3:3}
\end{VerbExample}

Amennyiben nem tudjuk biztosan, létezik-e a változó, azonban szükség van egy
alapértelmezett értékre, akkor a \texttt{\$\{NEV:-alapertelmezett ertek\}}
alakot használhatjuk, amikor is a változó eredeti értéke helyettesítődik be, ha
létezett és nem üres, ellenkező esetben a \texttt{:-} után megadott érték. Az
eredeti érték nem változik meg.

Majdnem ugyanezt teszi, azonban be is állít alapértelmezett értéket a
\texttt{:=} karaktereket tartalmazó kifejezés. A
\texttt{\$\{NEV:=alapertelmezett ertek\}} jelentése: amennyiben a \texttt{NEV}
változó nincs beállítva, vagy null (üres), akkor ezt az új értéket kapja meg a
változó, ezután az érték behelyettesítre kerül. Más esetben a változó értéke
helyettesítődik be.

Ritkán használjuk a következő változatot: az addott prefixű változók neveit 
helyettesíti be az \texttt{IFS} változó első karakterével elválasztva:
\texttt{\$\{!OP*\}}. Például a következő két név fog megjelenni:\\
\texttt{OPTERR OPTIND}.

A következőkhöz definiáljuk a \texttt{PELDA=almafa123456.txt} változót. A
\texttt{\$\{NEV/minta/szoveg\}} helyettesítésre szolgál, például:

\begin{VerbExample}
~ $ echo ${PELDA/a/T}
Tlmafa123456.txt
\end{VerbExample}

Ha viszont minden karaktert cserélni szeretnénk, az első perjelet meg kell
duplázni:

\begin{VerbExample}
~ $ echo ${PELDA//a/T}
TlmTfT123456.txt    
\end{VerbExample}

Ez nem valódi reguláris kifejezés, hanem csupán olyan, amelyet a shell a
fájlnevekhez használ:

\begin{VerbExample}
~ $ echo ${PELDA/a*1/K}
K23456.txt|
\end{VerbExample}

Változótömböket is használhatunk, ezek elemei kulcs-érték párokként viselkednek,
tehát beállíthatjuk a tömb indexét is az egyes elemekhez, viszont használhatjuk
a bash által alapértelmezetten válaszottt indexet is.

Használatuk:

\begin{VerbExample}
TOMB=()            # üres tömb létrehozása (előző érték törlésével)
TOMB=("alma"
      2
      3
      "Ez egy elem"
      T)          # tömb elemeinek felsorolása kulcs megadása nélkül (1)
TOMB[43]="alma"   # tömb adott elemének megadása (2)
echo ${TOMB[43]}  # Adott indexű elem kiírása (3)
echo ${TOMB[@]}   # Összes elem kiírása (4)
for ertek in ${TOMB[@]}
do
   echo $ertek
done              # Összes elem kiírása (5)
\end{VerbExample}

Az értékadásra az első eset az ideális, mivel ilyenkor biztosan nincsen ütköző
kulcs, vagyis biztosan nem írtuk felül egyik korábban beállított értéket sem. Ha
egy tömb elemeit akarjuk megadnim, használhatjuk a második eset szerinti
változatot is, ám ekkor ellenőriznünk kell, hogy az adott kulcs még nincsen
használatban, tehát a teljes tömb megadásakor nem célszerű ezt használnunk. Ám
egyes szkriptekben, ahol például egy ciklus közepén kell változtatni az értéket,
már nincs is lehetőségünk másfajta kódot írni. A harmadik eset mutatja, hogyan
tudunk egy adott indexű elemre hivatkozni, a negyedikben pedig az összes értéket
ömlesztve kiírni. Amikor ténylegesen felhasználjuk az elemeket, már inkább egy
for ciklus a célszerű, ahogy az ötödik esetben láthatjuk.


% Local Variables:
% fill-column: 80
% mode: latex
% End:
