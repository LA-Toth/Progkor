\chapter{Reguláris kifejezések}%
\label{cha:regex}

A reguláris kifejezésekkel (regular expressions, regex)
valamilyen szöveg keresésekor vagy helyettesítésekor
találkozhatunk. Ha egy szöveget (mintát) keresünk egy másik szövegben, ezt
mintaillesztésnek nevezzük. A helyettesítéskor az illeszkető mintát cseréljük le
egy adott szövegre, mely tartalmazhatja a minta egyes részeit, vagy akár a
teljes mintát is.

A reguláris kifejezések alatt a modern változatot értjük (lásd a regex(7)
kézikönyvlapot, melynek egy ``nyersfordítása''
\aref{sec:regex-man7}.\ alfejezetben olvasható). A régi, elavult
változatban például a zárójelek, plusz jel még normál karakternek számít, így
használja a \texttt{sed} program is.


\section{A reguláris kifejezések szerkezete: regex(7) kézikönyvlap}
\label{sec:regex-man7}

\emph{Ezen alfejezet a \emph{regex(7)} kézikönyvlap fordítása, amely itt-ott
 kibővült a más lapokon található információkkal.}

Egy (modern) reguláris kifejezés egy vagy több nem üres ágra (branch),
osztható, melyek között függőleges vonal (a `|' karakter, ``pipe''
karakter\footnote{Azért ``pipe'', mert ezzel lehet felépíteni a
parancsértelmezőben a több parancsból álló ``csővezetéket'', ami angolul
``pipe-line''}) áll. A kifejezés (vagy minta) akkor illeszkedik az adott
szövegre, ha az ágak valamelyike illeszkedik rá. Az ágakat a felsorolásuk
sorrendjében vizsgálja az adott program.

Egy ág több darabra (piece) oszlik, mely illeszkedik az elsőre, majd a
másodikra, stb. Egy darab több atomra osztható, melyek bármelyikét követheti
pontosan egy darab `*', `+', `?' vagy határokat jelölő kifejezés.

Egy atom utáni csillag (`*') karakter illeszkedik nulla vagy több (az adott)
atomból álló sorozatra. Ha pluszjel (`+') áll, akkor egy vagy több egymás utáni
(ismétlődő) illeszkedést jelöl. A kérdőjel (`?') nulla vagy egy illeszkedést
jelöl.

Egy határokat jelölő kifejezés (azaz intervallumot jelölő kifejezés) nyitó
kapcsoszárójellel (`\{') kezdődik, amit egy nemnegatív egész szám követhet, ami
után állhat egy vessző, illetve a vessző után egy másik nemnegatív szám, végül a
zárójel párja (`\}'). Tehát röviden $\{n,m\}$ alakű melyből bármelyik szám
elhagyható. Továbbá a $0 \le n, m \le 255$ egyenlőtlenségnek teljesülnie kell. A
\{n\} alak pontosan $n$ ismétlést jelent az $\{n,\}$ alak meg legalább $n$
ismétlődést jelöl.

Egy atom lehet egy reguláris kifejezés, mely körül zárójelpár, (`()'-ból) áll
(ez a regex egy illeszkedésére - illeszkedik), vagy egy üres `()', (a nulla
hosszúságú sztringet, null stringet)jelzi, vagy szögletes zárójeles kifejezés
(bracket expression) (lásd lejjebb), egy `.' (pont, mely pontosan egy, bármilyen
karaktert jelöl), egy `\^{}', (mely egy nullszringet jelöl a sor elején), egy
'\$' (mely a sorvégi nullsztringet jelöli), egy `\textbackslash' amelyet a
`\^{}.[\$()|*+?\textbackslash' karaktererek egyike követ (úgy illeszkedik az
adott karakterre, mintha az nem lenne speciális karakter, hanem a szöveg része
lenne), egy `\textbackslash' bármely egyéb karakterel együtt (ami az adott
karakterre illeszkedve, mintha a `\textbackslash' ott sem lenne), vagy bármely
más, nem speciális karakter (az adott karakterre illeszkedik). Továbbá ha egy
`\{' után nem áll számjegy, akkor a `\{' egy közönséges karakternek számít, nem
pedig egy intervallum elejét jelöli. Nem szabad egy regex végére `\textbackslash'-et
tenni.

Egy szögletes zárójeles kifejezés karakterek sorozata, mely körül `[]'
áll. Általában ez a benne álló karakterek valamelyikére illeszkedik (de lásd
alább). Ha a lista a `\^{}' karakterrel kezdődik, akkor bármely, a listában fel
nem sorolt karakterre illeszkedik (de lásd alább). Ha két karakter között `-'
áll, akkor az a két karakter között álló karakaterek sorozatának rövidítése, a
két adott karaktert is beleértve. Tehát például a `[0-9]' az \textsc{ascii}
kódtábla szerint bármely tizedes számra illeszkedik. A `[a-c-e]' forma viszont
illegális, mivel nem megengedett a tartományok végpontjainak megosztása. Ezen
tartományok erősen környezetfüggőek, úgyhogy minden
(portolható\footnote{program, több eltérő rendszeren is futtatni kívánt, és
lefordítható, futtatható, valamint minden rendszeren azonos módon viselkedik})
programnak kerülnie kell ezeket.

Egy közönséges `]' karakter jelölése úgy lehetséges, ha a lista elején áll (egy
esetleges `\^{}' karakter után). Egy közönséges `-' jelölése a listában annak
valamely szélére helyezésével tehető meg, vagy pedig akkor, ha tartomány
végpontjaként adjuk meg. Ha egy tartomány kezdőpontjaként kell szerepelnie,
akkor `[.' és `.]' közé zárva megtehetjük ezt. Ezen néhány kivételtől és a
következő bekezdésben szerepelőktől eltekintve az összes karakter, például a
`\textbackslash' is elveszti speciális jelentését egy bracket expression-ban.

% FIXME 
Egy ilyen zárójeles kifejezésben egy, vagy több, a `[.' és `.]' zárójelpár
között szereplő karakter olyan, mintha egyetlen karakter lenne a bracket
expression-ben. Ha több karakterből áll, akkor azok sorozatára
illeszkedik. Tehát például a `[[.ch.]]*c' RE (regex) a chchchcc' első 5
karakterére illeszkedik. Megadható ekvivalencia-osztály is,
lásd regex(7).

% TODO: befejezni

\section{A reguláris kifejezések alapjai és a grep}

\Aref{sec:regex-man7}.\ fejezetben a szabályos kifejezések (regex) pontos
definíciója szerepel. A következőkben néhány példán keresztül láthatjuk
ugyanezt.

Vegyük a legegyszerűbb mintát, például az `almafa' karaktersorozatot. Már ez is
egy reguláris kifejezés, hiszen azt mondja, hogy egymás után szerepelnek az `a',
`l', `m' stb.\ karakterek, mindegyik pontosan egyszer. Ha kicsit változtatunk
rajta, példáulaz `almaf.' változatot adjuk meg, akkor az olyan szövegre
illeszkedik, amely `almaf' karaktereket egymás után, ebben a sorrendben
tartalmazza, és utána van még egy karakter, amely bármelyik lehet. Az esetleges
továbbiakkal nem foglalkozunk, lehet, hogy vannak, azonban lehet, hogy
nincsenek, ha ez a szöveg vége. Tehát a pont (`.') egyetlen karaktert jelöl,
amire semmiféle megkötés nincsen.

Tegyük fel, hogy az ``.sh'' végződésű állományokban keresünk bizonyos
függvényeket (szintaxisa: \texttt{function név()}, ahol a function nem
kötelező). A következő néhány példa erről szól.

Először keressük a ``main'' nevű függvényt:

\begin{VerbExample}
grep 'main()' *.sh
\end{VerbExample}

\noindent de nem találunk ilyet, mert a zárójel előtt van számjegy is (a név
legyen például 'main1'). A számjegy utáni
résszel nem akarunk foglalkozni, így a zárójelekkel sem. Ekkor a minta a
következő lesz: `main.'. Ez sajnos még nem tökéletes, hiszen a main utáni
zárójelre is illeszkedik, a számjegyeket valahogy jelölnünk kell. Itt is működik
a bash-ben meglévő intervallum (karakterhalmaz) megadása:

\begin{VerbExample}
grep 'main[0-9]' *.sh
\end{VerbExample}

\noindent ami már megfelelő. De hogyan is működik ez a megadás? A szögletes
zárójelpár
között egy vagy több karakternek kell szerepelnie, ha egymás utáni az adott
kódolás esetén, akkor lehet rövidíteni: a legelső és az utolsó közötti elemek
helyett egyetlen kötőjel (mínuszjel) elegendő. Persze nem mindig szükséges,
például `[abc]' helyett írható `[a-c]', mely olvashatóbb, de nem rövidít semmit
sem. Lehet több intervallum is benne, például a 'g' nélküli összes angol
kisbetű: `[a-fh-z]'.

Ha a példában a számjegy állhat hátrébb is, akkor jelezni kell, hogy előtte
szerepelhet más karakter is, de nem kötelező lennie. Ha az adott karakter mögött
csillagot (`*') van, akkor e karakter bárhányszor (0-szor vagy többször)
ismétlődhet.

\begin{VerbExample}
 grep 'main.*[0-9]' *.sh
\end{VerbExample}

Sajnos ebben benne van  a zárójel is. Ha a függvény törzse a nevével egy sorban
van, akkor hibázhat. Így jobb megoldás szűkíteni a neveket:

\begin{VerbExample}
grep 'main[a-zA-Z_]*[0-9]' *.sh    
\end{VerbExample}

Ha bizonyos karaktereket nem szeretnénk megengedni, rajtuk kívül azonban
mindegyiket, akkor is hasonló jelölést kell alkalmaznunk. Például a nem
kisbetűkre: `[\^{}a-z]'. Az egyetlen eltérés, hogy az elején van egy `\^{}', ami
a tagadást jelenti.

A `]' és `-' karakterek megadása úgy lehetséges, hogy az első a legelső karakter
az esetlegesen előforduló `\^{}' karakter után, a másik pedig az utolsó,
például: `[\^{}]abe-h-]' vagy `[ab-]'. 

Ezek alapján egyértelmű a következő példa jelentése:\\

\begin{VerbExample}
grep 'main[^][().{}]*[0-9]' *.sh
\end{VerbExample}

\noindent bár elsőre félreérthető, hiszen A `][' rész is a felsorolt karakterek
között van.

Nem csak a `*' használható az ismétlődés jelölésére, hanem a kérdőjel (`?') és a
pluszjel (`+') is. Az előbbi azt jelenti, hogy az előtte álló karakter (vagy
más, lásd később) \emph{legfeljebb} egy példányban fordul elő, míg az utóbbi, a
`+', pedig azt, hogy \emph{legalább} egyszer előfordul, azaz
ismétlődik. Lényeges, hogy a régi reguláris kifejezések (basic regex) számára a
`+' és a gömbölyű zárójelek: `(' és `)' is egyszerű karakterek, a modern (vagy
kiterjesztett) reguláris kifejezésekben viszont már speciális jelentéssel
bírnak. A \texttt{grep} a régit, az \texttt{egrep} pedig a modern
(\emph{e}xtended) változatot használja. Persze mindkettő ismeri a másikat is,
csak akkor a kérdéses karakterek előtt kell egy backslash
(`\textbackslash'). Következzék néhány példa, melyekben a `main' után legfeljebb
egy (első példa), illetve legalább egy (többi) számjegy állhat, melyet
zárójelpár követ.

\begin{VerbExample}
grep 'main[0-9]?()' *.sh
grep 'main[0-9]\+()' *.sh
egrep 'main[0-9]+\(\)' *.sh    
\end{VerbExample}

Az \texttt{egrep} esetén azért kellett a backlash (`\textbackslash'), mert az
megváltoztatja az utána álló karakter jelentését (azaz ``escape karakternek''
számít). Ilyen létezik a shell esetén is, például a \verb|$'\n'| az újsor
karaktert jelöli, a szóköz előtt álló `\' pedig a szóközt a fájlnévben, vagyis a
bash nem tekinti paraméter-elválasztó karakternek.  A zárójelek pedig
csoportosításra szolgálnak, például az ``alma''
karaktersorozat ismétlődésének jelölése a következő: `(alma)+'. Ha még
hozzávesszük a pipe karaktert (`|') is, akkor a zárójelen belül felsorolhatunk
lehetőségeket is, például az ``alma'', vagy pedig a ``körte'' ismétlődik:
`(alma|körte)+'.

Térjünk most vissza \aref{sec:filters-grep}.\ fejezetben mutatott példához,
amelyben azon felhasználók azonosítóit irattuk ki, akik \texttt{bash}-t
használnak:

\begin{VerbExample}
grep /bash$ /etc/passwd | cut -f1 -d:
\end{VerbExample}

\noindent Irassuk ki azokat is, akik akik az \texttt{sh}-t használják:

\begin{VerbExample}
egrep '/(bash|sh)$' /etc/passwd | cut -f1 -d:
\end{VerbExample}

\noindent  Mivel a végződés azonos, jelölhetjük úgy is, hogy a ``ba'' karakterek
legfeljebb egyszer szerepelnek:

\begin{VerbExample}
egrep '/(ba)?sh$' /etc/passwd | cut -f1 -d:
\end{VerbExample}

A `\$' nem feltétlenül a minta utolsó karaktere, pusztán a sor (vagy a szöveg)
végét jelzi, arra illeszkedik. Ugyanez igaz a szöveg elejére illeszkedő `\^{}'
karakterre is. Például a sor végén álló ``bash'' szót tartalmazó sorokat
szeretnénk kiírni, de előfordulhat, hogy a bash után van még perjel (`/') is, és
ezek a sorok is érdekesek. Az viszont már nem fontos, hogy a perjel után mi
áll:

\begin{VerbExample}
grep '/bash($|/)' /etc/passwd    
\end{VerbExample}

\subsection{Karakterosztályok}

A reguláris kifejezések esetén lehet karakterosztályokat is használni, például a
``lower'' a kisbetűknek felel meg. Maradjunk egy kicsit ennél az
osztálynál. Angol nyelvi beállításnál ez megegyezik az ``a-z'' intervallummal,
ám magyar környezetben ebben már benne vannak az ékezetes betűink is. Hogy lehet
használni: `[:lower:]' megegyezik az `a-z' kiírásával (angol változat esetén),
ha viszont rendesen `[a-z]'-t szeretnén kírni, akkor már kell a külső
zárójelpár: `[[:lower:]]'. Ez mindenképpen szükséges, hiszen tegyük fel, hogy a
nyelvi beállításoknak megfelelő kisbetűkön felül meg szeretnénk engedni az `A'
és a `B' nagybetűket is, melyet az alábbi módon jelölhetünk: `[[:lower:]AB]'.

A nyelvi beállításokat az \texttt{LC\_ALL} vagy \texttt{LC\_CTYPE} változókkal
definiálhatjuk, erre példa a ``C'', ``POSIX'', vagy éppen magyar nyelv esetén a
``hu\_HU'' illetve ``hu\_HU.UTF-8''.

Van még két hasonló karakterosztály, az ``alpha'', mely az összes betűt
jelöli, az ``upper'', mely a nagybetűket, és igaz a következő: az `[[:alpha:]]'
ugyanazt a karakterhalmazt jelenti, mint a `[[:lower:][:upper:]]'. Ez igaz az
angol és a magyar nyelvi beállítások esetén, de egyes nyelveknél előfordulhat
olyan betű, ami sem nem nagy, sem nem kicsi.

Ha még hozzávesszük a számjegyeket is, amit a ``digit'' karakterosztály jelöl,
akkor a `[[:digit:][:alpha:]]' jelöléssel az összes betűt és számjegyet
megadtuk. Azonban ez is egy gyakran használt kifejezés, ezért e csoportra is van
egy önálló osztály, az ``alnum'' (alphanumeric characters), így egyszerűbb és
olvashatóbb is ezt írni: `[[:alnum:]]'.

A ``digit'' a decimális számokat jelöli, azonban a hexadecimális számjegyekre is
van egy jelölés, az ``xdigit''. Tehát a következők azonosak: `[0-9a-fA-F]',
`[[:xdigit:]]' stb.

A szavak közt előforduló karakterekre (szóköz, tabulátorkarakter, újsor karakter
stb.) is van két karakterosztály. Az egyik a ``blank'', amely csupán a szóközt
és a tabot tartalmazza, azonban a ``space'' többet. A pontos jelentése csak a
``C'' és ``POSIX'' környezetben rögzített, ekkor az újsor (\verb|\n|), a
kocsi-vissza (carriage return, sor elejére ugrás, \verb|\r|), a függőleges tab
(\verb|\v|), és a form-feed (\verb|\f|) karakterek is benne vannak.

Hogyan lehet eldönteni, hogy valamilyen szöveg egyetlen számot tartalmaz? Tegyük
föl, hogy a ``SZAM'' változó tartalmazza ezt és a \texttt{grep} paranccsal
ellenőrizzük. Elegendő a státuszát megtekinteni, ezért használhatjuk a
\texttt{-q} opciót, aminek hatására nem ír ki semmit sem, viszont 0 státusszal
jelzi, ha illeszkedik a minta a szövegre, nem nullával, ha nem (azaz
sikertelenül futott le a program).

Néhány lehetséges megoldás nemnegatív egész számok esetén:

\begin{VerbExample}
echo "$SZAM" | egrep -q '^(0|[1-9][[:digit:]]*)$'
echo "$SZAM" | egrep -q '^(0|[1-9][0-9]*)$'
\end{VerbExample}

Negatív számok esetén a nullának nem lehet előjele, de a többinek igen: egy
legfeljebb egyszer előforduló negatív előjel is szerepel:
\begin{VerbExample}
echo "$SZAM" | egrep -q '^(0|-?[1-9][[:digit:]]*)$'    
\end{VerbExample}

Egyes esetekben a pozitív számoknak is lehet előjelük. A kötőjel (negatív
előjel) a felsorolásban az utolsó, hiszen így nincs speciális jelentése\ldots

\begin{VerbExample}
echo "$SZAM" | egrep -q '^(0|[+-]?[1-9][[:digit:]]*)$'
\end{VerbExample}

Most engedjük meg a lebegőpontos számokat is, ahol az egészrész végén van egy
pont (`.') karakter, majd utána nem csupa nulla számjegyekből álló sorozat
szerepel, így a végét kell bővíteni:

\begin{VerbExample}
echo "$SZAM" |
   egrep -q '^(0|[+-]?[1-9][[:digit:]]*)\.(0|[[:digit:]]*[1-9])$
\end{VerbExample}

Ezt lehetne tovább bővíteni, azonban ez alapján már könnyen definiálható például
az az eset, amikor egész értéknél nem szükséges a tizedesjegy és tizedespont
megadása, valamint egynél kisebb abszolútértékű szám esetén nem kell a
tizedespont előtt álló 0 sem.



\section{A reguláris kifejezések és a sed}
\label{sec:regex-sed}

A \texttt{sed} esetén is ugyanolyan reguláris kifejezéseket lehet használni,
mint a \texttt{grep} esetén. A modern regex a \texttt{-r} opcióval használható,
érdemes ezt mindig megadni.

A karakterhelyettesítést (\texttt{y/lista1/lista2/}) nem érinti, viszont a
szöveghelyettesítést már igen. Sőt, a \texttt{sed} alkalmas bizonyos sorokra más
műveletet végrehajtani, mint a többire, szkriptelhető.

 A zárójelezett kifejezésekre lehet hivatkozni,
legalábbis az első kilencre a következő módon: \verb|\N|, ahol \verb|N| jelzi,
hogy az zárójelezett kifejezéshez hanyadik nyitózárójel tartozik. A
`(a(b(c)d(e)))' kifejezésben az `e' körül a negyedik zárójelpár áll.

Például az ugyanazzal a három karakterrel kezdődő és végződő sorokból
elhagyhatjuk ezt a részt. Csak a középső marad meg:
\begin{VerbExample}
sed -r 's/^(...)(.*)\1$/\2/'
\end{VerbExample}

\noindent az elejét elhagyjuk, ekkor másképp kell zárójelezni:

\begin{VerbExample}
sed -r 's/^(...)(.*\1)$/\2/'
\end{VerbExample}
most a végét hagyjuk el, az első három karaktert már a második zárójelpár
tartalmazza:

\begin{VerbExample}
sed -r 's/^((...).*)\2)$/\1/'
\end{VerbExample}

Ha minden harmadik karakter után egy szóközt szeretnénk beszúrni, akkor egy
lehetőség az alábbi:

\begin{VerbExample}
sed -r 's/(...)/\1 /'
\end{VerbExample}

\noindent de ez nem elegáns, mivel felesleges zárójelezni. Sőt, alap reguláris
kifejezéssel is működik:
%\utbeh\verb|sed -r 's/.../& /g'|\\

\begin{VerbExample}
sed 's/.../& /g'
\end{VerbExample}

Itt az \verb|&| karakter a teljes, mintára illeszkedő szöveget jelöli, jelen
esetben ez 3 karakter hosszú.

% Local Variables:
% fill-column: 80
% mode: latex
% End:
