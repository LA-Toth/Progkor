\chapter{Egyebek}%
\label{cha:others}

%E fejezetben további, a programozási környezet tárgyhoz szigorúan nem tartozó
%programok és példák szerepelnek.



\section{Az \texttt{xargs} parancs}

Az \texttt{xargs} tekinthető speciális szűrőnek, ugyanis a szabványos bemenetére
-- soronként -- érkező fájlneveket a paraméterében megadott programnak adja
át. Ha nincs megadva, akkor ez az \texttt{echo}, amely a bash beépített parancsa
is és egy külső program is. Mindkét változat elfogadja a \texttt{-n} opciót,
amely nem ír újsor karaktert, valamint a \texttt{-e} opciót is, amellyel az
escape szekvenciákat értelmezi (például: \texttt{echo -e 'a\t'}).

Az \texttt{xargs} egy hasznos opciója a \texttt{-o}, melyet általában nem
használunk, ám amikor interaktív programot adunk meg paraméterként, akkor
elengedhetetlen, ugyanis e program bemenete a \texttt{/dev/tty} lesz, vagyis
amin éppen dolgozunk.

Egy másik opciója, amelyre szükségünk lehet, a \texttt{-t}, amely a
hibakimenetre írja ki, mit fog most lefuttatni.


\begin{VerbExample}
echo almafa | xargs rm            # Az almafa fájl törlése
\end{VerbExample}




\section{A \texttt{find} parancs}

A \texttt{find} paranccsal kereshetünk egy adott könyvtárstruktúrában.

\begin{VerbExample}
find könyvtár feltételek
find feltételek            # csak GNU find, kerüljük, helyette inkább:
find . feltételek
\end{VerbExample}

A \texttt{find} alapesetben az aktuális könyvtár alatt keres, ám megadható az
is, hogy ezt melyik alkönyvtárban tegye. Bár Linuxon működik mindkét forma,
régebbi Unixokon, valamint a mai \textsc{bsd} rendszereken a könyvtár nem
hagyható el, mindig meg kell adni.

A legfontosabb opciói:

\begin{tabular}{ll}\toprule
Opció  & jelentés\\\midrule
-name X & Az ``X'' nevű könyvtárbejegyzéseket keresi\\
-iname X & mint az -iname, csak kis- és nagybetűk azonosak\\
-tpye X & Adott típusú könyvtárbejegyzés\\
-maxdepth n & Legfeljebb n mélységben keres\\
-exec program [params] ; & Adott program futtatása\\%
\bottomrule
\end{tabular}

További opciók is léteznek, ezek közül néhányat a példák során vizsgálunk meg.

A \texttt{/etc} környvtárban keressük az összes ``pas'' karakterrekkel kezdődő,
``d''-re végződő fájlt vagy könyvárat, a második példűban pedig már csak a
fájlokat.

\begin{VerbExample}
panther@Artanis ~ $ find  /etc/ -name 'pa*d'
/etc/pam.d
/etc/pam.d/passwd
/etc/passwd
/etc/paths.d
panther@Artanis ~ $ find  /etc/ -name 'pa*d' -type f
/etc/pam.d/passwd
/etc/passwd
\end{VerbExample}

A \texttt{-name}, továbbá a \texttt{-iname} argumentuma is egy ``glob-pattern''
elnevezésű kifejezés, amelyhez hasonlót használ a shell is. Itt minden karakter
önmagát jelenti, kivéve a kérdőjel, amely pontosan egy tetszőleges karaktert
jelöl, valamint a csillag, amely legalább nulla tesztőleges karaktert. 

A \texttt{-type} argumentuma leggyakrabban az \texttt{f} (reguláris fájl), és a
\texttt{d} (könyvtár), azonban további lehetőségek is vannak, blokkos
eszközfájlra a \texttt{b}, karakteresre a \texttt{c}, szimbolikus linkre pedig a
\texttt{l} (kis L, nem pedig egy). Még gyakran találkozhatunk socket fájllal
(\texttt{s} a típus jele), és előfordulhat FIFO is (\texttt{p}). 

Fájlok törlésére is több lehetőségünk van, az egyik az \texttt{rm} parancs
meghívása, a másik pedig egy \texttt{find} opció, a
\texttt{-delete}. Jellegzetes felhasználása átmeneti fájlok törlése:

\begin{VerbExampleNum}
find . -type f -iname '*.tmp' -exec rm {} \;
find . -type f -iname '*.tmp' -exec rm '{}' \;
find . -type f -iname '*.tmp' -exec rm '{}' '+'
find -d . -type f -iname '*.tmp' -exec rm '{}' \;
find . -type f -iname '*.tmp' -delete
find . -type f -iname '*.tmp' | xargs rm
\end{VerbExampleNum}

Ezek nagyon hasonló parancsok, az első kettő pontosan ugyanaz, az aposztróf
használata nem kötelező. Viszont ez hibalehetőség is, ugyanis ha a zárójelpár
nem üresen áll, a bash által értelmezhető is lehet: \texttt{\{a,b\}} kifejtés
után \texttt{a b}, és ezek teszőlegesen egymásba ágyazhatók (két vagy több
paramétert esetenként könnyebb így megadni).

A kapcsoszárójelpár helyére a \texttt{find} behelyettesíti a megtalált
könyvtárbejegyzést, és pontosan egyet az első két esetben. A futtatandó parancs
végét egy -- a \texttt{find}-nak átadott -- pontosvessző jelzi. A harmadik
változatnál az \texttt{-exec} utolsó argumentuma viszont egy pluszjel, amely
hatására a lehető legtöbb argumentumot próbálja meg átadni a find az adott
programnak (\texttt{-exec} első argumentuma).

A negyedik változatnál a \texttt{find} a \texttt{-d} opció hatására mélységi
bejárást végez az adott könyvárstruktúrán, az ötödik pontosan ugyanezt teszi,
csupán nem kell külső programot meghívni a törlésre, és a mélységi bejárás
automatikus.

A hatodik esetben a \texttt{find} parancs a szabványos kimenetre írja a
megtalált fájlokat, melyet az \texttt{xargs} feldolgoz és az összeset egyszerre
átadja az \texttt{rm}-nek. Hatása majdnem ugyanaz, mint a harmadik változaté. A
\texttt{find} használatakor nem is szükséges, más egyéb programok, szkriptek
esetén hasznos. Van azonban egy nagy különbség a harmadik változathoz képest,
nevezetesen az, hogy ha a megtalált állományok között van szóközt tartalmazó,
akkor az \texttt{xargs} ott szétvágja, vagyis nem egy, hanem több paramétert ad
át a mögötte álló programnak -- amely itt éppen az \texttt{rm}.



% Local Variables:
% fill-column: 80
% mode: latex
% End:

