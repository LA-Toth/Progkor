\chapter{Szűrők}%
\label{cha:szűrők}

Egy Unix-hoz hasonló rendszeren, amilyen a Linux is, számtalan program
található, amelyek együttes használatával lényegében bármilyen feladatot
elvégezhetünk. Ezek alapvetően szövegfeldolgozásra -- és számításra --
készültek, ugyanis kezdetben nem volt grafikus felület, azonban ma már másra is
használható a parancssor, például egyszerűbb képfeldolgozásra (kivágás,
összeillesztés, pixelek megváltoztatása stb.).

\Aref{cha:shell}.\ fejezetben látott \texttt{grep} és \texttt{cut}
parancsokhoz hasonló eszközök valamilyen bemenetet olvasnak és kimenetükre
írnak, miközben azt a legtöbb esetben módosítják is. Tulajdonképpen megszűrik
azt, ezért lett összefoglaló nevük: \emph{szűrők} (filters).

Bár a fenti példák esetén egy másik program kimenete volt e szűrők bemenete, nem
csak innen olvashatnak adatokat -- legtöbbször sorokat --, hanem fájlokból
is. Ha paraméterként nem adunk meg fájlnevet, akkor viszont a \emph{szabványos
  bemenetről} olvasnak, ami lehet a billentyűzet, de lehet egy másik program
kimenete is. Az angol elnevezése \emph{standard input} vagy röviden stdin. Ahova
e szűrők írnak, az a \emph{szabványos kimenet}, előzőek szerint standard output
vagy stdout. Végezetül az esetleges hibákat a \emph{szabványos hibakimenetre}
(standard error, stderr) írja a program.

A kimeneteket át lehet irányítani vagy pedig egy másik program
bemenetére lehet csatolni, az előbbivel a következő alfejezet foglalkozik. Az
utóbbi esetben a két program egy csővezetéket (pipeline) alkot, hiszen a
korábban megadott program kimenete a másik bemenetére jut, mintha egy csőben
lennének, máshovan nem kerülhet adat. A programok közé egy ennek megfelelő jelet
(angolul pipe -- cső -- karakter), a  \verb,|, karaktert kell tenni. A
csővezeték lehet hosszabb is, melynek minden darabja a feladat egy kis szeletét
oldja meg, mégis az egész eredménye egy esetlegesen bonyolult feladat megoldása.

Most a következőkről lesz szó: átirányításról \aref{sec:filters-redirect}.\ %
alfejezetben, majd ezek után a lényegesebb szűrőkről, különös tekintettel a
\texttt{grep} és \texttt{sed} programokra \aref{sec:filters-grep}.\ és
\aref{sec:filters-sed}.\ alfejezetekben.


\section{Átirányítások}
\label{sec:filters-redirect}

A programok kimenetét át lehet irányítani, egyrészt a szabványos kimenet és
hibakimenet átirányítható a másikba, másrészt fájlba is kerülhet a kiírt szöveg
(adat).

\begin{VerbExampleNum}
program  >fájl     # fájlba írás felülírással
program >>fájl     # fájlba írás végére fűzéssel
program  >&2       # stdout átirányítása stderr-re
program 1>&2       # szintén, csak a kimenet száma is ki van írva
program 2>&1       # stderr az stdout-ra
program 2>fájl     # hibák (stderr) fájlba írása
program >| fájl    # fájlba írás, kényszerített változat
\end{VerbExampleNum}

Figyeljünk arra, hogy ha létezik a fájl, akkor az első esetben felülíródik a
tartalma. Ekkor a bash nulla hosszúságúra csonkolja a fájlt, s csak ezután kapja
meg a program. Ezért nem megy az, hogy egy szűrő beolvas egy fájlt, majd pedig
egyből vissza is írja ugyanazt a fájlt, hiszen még az olvasás előtt nulla
hosszúságú lesz.

Ezért létezik egy bash beállítás, amely véd a \emph{véletlen} felülírás ellen,
amilyen az előző is, ennek beállítása:
\begin{VerbExample}
shopt -so noclobber  
\end{VerbExample}

Minden kimenetnek van egy azonosító száma, a szabványos kimeneté az egyes, a
hibakimeneté pedig a kettes. Az átirányításokban ezt kell használni, az alapeset
az, amikor a kimenetet irányítjuk át. Ha a másikra, és nem a fájlba kerül az,
amit a program kiír, akkor a nagyobb jel (\verb|>|) után egy és jel (\verb|&|)
szerepel, mely után azonnal jön az a szám, amelynek megfelelő kiemetre át
akarjuk irányítani, a nagyobb jel előtt közvetlenül pedig az a szám, amely
kimenetet átirányítjuk. A parancsot elhagyva a lehetséges átirányítások ezek
szerint az alábbiak:

\begin{VerbExample}
 >&2   # stdout stderr-re (az alapértelmezett: sdout)
1>&2   # stdout stderr-re (explicit módon megadva a szabványos kimenetet)
2>&1   # stderr stdout-ra
\end{VerbExample}

Természetesen ugyanígy működik a fájlba íratás is. A kérdés már csak az, hogy mi
történik, ha egyszerre átirányítjuk mindkét kimenetet. Ha mindkettőt különböző
fájlba, akkor nincsen semmi különleges benne. Ha viszont az egyiket egy fájlba
irányítjuk, akkor számít a sorrend: ha valamelyiket elirányítottuk, akkor az már
úgy marad, részletesebben lásd lejjebb.

\begin{VerbExampleNum}
parancs  >&2 2>filename
parancs 2>&1  >filename
parancs 2>filename >&2
\end{VerbExampleNum}

Az első kettő csak az stdout-stderr helyében tér el, ezért tekintsük az első
esetet. A bash az első átirányításkor -- stdout az stderr-re -- még nem tudja,
mi fog történni a hibakimenettel, ezért \emph{ attól függetlenül, mindenképpen},
az stderr-re kerül az stdout kimenete. Az \emph{eredeti} hibakimenet viszont
fájlba íródik.

Ezek után már könnyen kitalálható, mi történik a harmadik esetben. A szabványos
hibakimenet egy fájlba kerül, és ezután irányítjuk át a kimenetet a
hibakimenetre, amiről már tudja a bash, hogy hova kerül, ezért az is beleíródik
a fájlba.

\section{Alapvető szűrők}
\label{sec:filters-simple}

Egy telepített rendszeren nagyon sok különböző kis programocska lehet, melyeknek
egy része mindennap használatos (például a \texttt{grep}, egy másik részére
ritkán bár, de szükségünk lehet (például az \texttt{nl} és az \texttt{od}),
másokat lehet, hogy sosem használunk. Ez  a rész a gyakori szűrőkkel
foglalkozik.


\subsection{Fájl kiiratása}
\label{ssec:filters-cat}

A legegyszerűbb szűrő az a program, amely a paramétereiben felsorolt fájlokat
összefűzi, egymás után kiírja a tartalmukat. Bár egy szövegfájlban az összes sor
végén áll egy újsor karakter, ha az utolsó sorban valamiért még sincsen, majd
ezt és egy másik fájlt is kiiratunk, akkor e fájl utolsó sora és a következő
paraméterben madott állomány első sora a képernyőn egyetlen sorba kerül --
feltéve, ha kifér. E program a \texttt{cat}, amely a konkatenálás  (concatenate)
szóból kapta a nevét. Ha elhagyjuk a paramétereit, akkor a standard inputról
olvassa be a sorokat, és változatlan formában kiírja, mely éhány esetben hasznos
lehet. Most pedig lássunk pár példát.

\begin{VerbExampleNum}
cat >file               # billentyűzetről új fájlba írás
cat f1 f2 f3 >f4        # több fájl összefűzése eggyé
cat f1 f2 f3 | program  # több fájlt összefűzve egy másik programnak
\end{VerbExampleNum}

Az első eset editor nélküli fájlba írást tesz lehetővé, így nem kell megvárni,
amíg a szerkesztő betöltődik a memóriába. A második például akkor hasznos, ha
egy \texttt{iso} fájlt szétdarabolva tesz fel valaki, hogy ne kelljen túl nagy
darabokat újra letölteni, ha a fájl menet közben megsérül, valamint le lehessen
tölteni akkor is, ha a webszerver nem kezeli a nagy fájlokat (ha van, akkor 2
GiB ez a korlát, 
egy \textsc{dvd} lemez pedig 4.4 GiB méretű).  Ekkor csak össze kell fűzni
egyetlen állománnyá. A harmadik változat meg akkor lehet hasznos, ha sok
különböző fájlon kellene ugyanazt megtenni, és ráadásul az eredmény lehet
egyben, nem pedig külön-külön fájlonként.

Érdekességképpen mivel a program neve macskát is jelent, ezért ennek egy
módosított változatát is megírták, melynek kutya (\texttt{dog}) lett a neve, és
már weboldalakat is ki tud írni a szabványos kimenetére:\\
\texttt{dog http://www.kernel.org/}.

\subsection{A fájlok eleje és vége}

A következő két szűrő a fájlok elejét illetve végét írja ki, alapesetben tíz
sort, ám a megfelelő opciót megadva ettől eltérő számút is ki tud írni, valamint
lehetőséget nyújt arra is, hogy valahány sort ne írjon ki. A fájlok elejéről a
\texttt{head} olvas, végéről pedig a \texttt{tail}. A régi \textsc{unix}
rendszerek valamint a \textsc{bsd} család esetén is ezt a számot mint opciót
kell meadni, a Linuxokon futó (\textsc{gnu}) változat pedig elvárja, hogy a
\texttt{-n} opció után szerepeljen ugyanez az érték.

\begin{VerbExample}
head fnev       # első 10 sort
head -n10 fnev  # első 10 sort
head -n-2 fnev  # utolsó 2 sor kivételével mindent
tail fnev       # az utolsó 10 sort
tail -n8 fnev   # az utolsó 8 sort
tail -n+3 fnev  # a 3. sortól mindent
tail -c5        # utolsó 5 karakter 
\end{VerbExample}

Ezeket kombinálva a fájl bármely szelete megkapható. Arra érdemes figyelni, hogy
amennyiben nagyon sok sort kell feldolgozni, akkor már az első szűrő kimenete is
a lehető legrövidebb legyen. Például a 3. sortól a 8.-ig történő kiírás esetén
előbb az első 8 sort érdemes meghagyni, majd ebből tovább szűrni, nem pedig
először a 3.-tól kezdve mindent, majd pedig ebben az első 6 sort meghagyni,
ugyanis csak így lesz gyors.

Ha nem a \texttt{-n}, hanem a \texttt{-c} opciót használjuk, akkor a fájlok vagy
pedig a szabványos bemenet első illetve utolsó néhány bájtját tudjuk kiiratni.

\subsection{A sorok részei}

Nem csak a fájl bizonyos sorait irathatjuk ki, hanem soronként haladva azok
valamely darabját is. Ehhez nyújt segítséget a \texttt{cut} program, amellyel a
sorok valamely \emph{mezője} vagy oszlopa íratható ki. Az alapértelmezett
mezőelválasztó a tabulátor karakter, azonban megadható bármely másik
is. Ha az adott sorban ez egyszer sem szerepel, akkor a teljes sor alkotja az
első mezőt, illetve ha egymás után kétszeri is megtalálható, akkoris van köztük
egy mező, amely üres, így a \texttt{cut} alkalmatlan például arra, hogy
szóközökkel feltötött szövegben mezőket vágjon ki. Erre egy példa az \texttt{ls
  -l} utasítás kimenete, amely oszlopokra tagolható, tehát a karakterek
kivágásával feldolgozható lenne, ámde előre nem ismert az egyes oszlopok
szélessége. A különböző könyvtárakban kiadott \texttt{ls -l} kimenetek
összehasonlításakor szembeötlő a különbség: a tulajdonost és a csoportot jelző
oszlop nem fix szélességű, ráadásul a fájl módosítási idejének megjelenítési
formája is változó.

A \texttt{-d} után adható meg a mezőelválasztó, a \texttt{-f} után egy vagy több
mező, végül \texttt{-c} után a karakterek oszlopa (indexe). Akárhogy is adtuk
meg a mezőket vagy az oszlopokat, mindig növekvően fogja megjeleníteni, azaz a
\texttt{-f2,4,6} és a \texttt{-f6,2,4} hatása pontosan ugyanaz. Vesszővel
elválasztva lehet több indexet megadni, amely lehet egy hosszúságú, ahogy az
előbbi példákban szerepelt, valamint megadhatunk intervallumot is, a kezdő és a
végpontot kötőjellel (mínusz jellel) elválasztva. Amennyiben a kezdő vagy pedig
a végpont a sor első vagy utolsó karaktere, akkor nem szükséges kiírnunk a
számát.

\begin{VerbExample}
cut -f1        # első mező
cut -f-2,6     # második mezőig, valamint a 6. mező
cut -c7,32-    # a 6. karakter (oszlop) és a 32.-től
cut -f2 -d:    # a 2. mező; elválasztó a kettőspont
\end{VerbExample}



\subsection{Rendezés és ismétlődő sorok}

Számos esetben szükséges az azonos sorok kiszűrése vagy számlálása, amelyhez
elengedhetetlen a sorok rendezése, hiszen így az eredetileg $O(n^2)$ időt és
$O(n)$ tárat igénylő összehasonlítás már lineáris idejű és csupán konstans tárat
igényel. A rendezéshez a \texttt{sort} parancs használható, az ismétlődő sorok
szűrésére pedig az \texttt{uniq}.

A rendezés függ a nyelvi beállításoktól, például az
\texttt{LC\_ALL=hu\_HU.UTF-8} környezeti változó esetén az \textsc{utf-8}
kódolású fájlokat is helyesen dolgozza fel, valamint az 'a' betű után az 'á'
következik, míg \textsc{LC\_ALL=C} esetén nem jó helyen szerepelnek az ékezetes
karakterek. Sőt, nem csak ábécé szerint rendezhetünk, hanem számok nagysága
szerint, így nem a ``10'' után következik a ``2''. 

Mivel a sorok rendezése és az ismétlődő sorok elhagyása gyakori feladat, ezért
erre van egy külön opció is, a \texttt{-u}, így a \texttt{sort | uniq} helyett
elegendő \texttt{sort -u}-t írnunk, ami rádásul gyorsabban is fut.

Az \texttt{uniq} program is számos opcióval rendelkezik, a leggyakoribbak a
\texttt{-c}, amely az ismétlődő sorok előfordulási számát írja ki (1 ha csak
egyszer fordult elő), \texttt{-d} a többször szereplőket, végül \texttt{-u} az
egyedieket.

\begin{VerbExample}
sort            # abc szerint rendez
sort -n         # számok szerint
sort | uniq -c  # melyik sor hányszor szerepel
sort -r         # fordított irányú rendezés
\end{VerbExample}


\section{Sorokban szöveg és karakterek cseréje}
\label{sec:filters-replace}

Soron belül egy vagy több karaktert, vagy akár a teljes sort is le lehet
cserélni, a konkrét feladattól függően több lehetőség közül is
választhatunk. Karakterek cseréjére alkalmas a \texttt{tr}, valamint a
\texttt{sed} is, azonban hosszabb szöveget már csak a \texttt{sed} segítségével
cserélhetünk. Vannak további lehetőségek is (\texttt{awk} és valamilyen másik
szkriptnyelv használata is, például a Perl \cite{perl} is, de most ezekkel nem
foglalkozunk, az \texttt{awk}-ról is később lesz szó).

Manapság ha karakterekről valamint azokkal kapcsolatos műveletekről hallunk, a
legelső kérdés, amit meg kell vizsgálnunk, hogy éppen melyik karakterkódolásról
van szó, az adott eszköz boldogul-e az eltérő ábrázolással. Számunkra most az
\textsc{iso-8859-2} (latin2, egy bájtos) és az \textsc{utf-8} (több bájtos)
kódolások az érdekesek. A \texttt{sed} ismeri mindkettőt, a \texttt{tr} azonban
csak az egybájtos ábrázolásokkal képes működni.

A \texttt{tr} talán az egyetlen olyan szűrő, amely kizárólag a standard inputról
olvas szöveget, fájlt nem képes kezelni. Alapesetben két paramétert vár, két
karakterhalmazt, amelyek között a csere történik. A karaktereket egymás után
adjuk meg, így az első paraméterben megadott karakterek első elemét a másodikban
megadottak első elemére, másodikat a másodikra stb.\ cseréli le. Lehet törölni
is karaktereket, ekkor viszont nem kell a második paraméter, hiszen az első
halmaz elemeit kell törölni, a \texttt{-d} opció használatával.

A karekterhalmazok állhatnak egyesével felsorolt elemekből, valamint
intervallumokból
is, például az angol kisbetűket tartalmazó \texttt{[a-z]} halmazból. Szerepelhet
itt oktálisan (8-as számrendszerben megadott) karakter is \verb|\NNN| alakban,
például a szóköz, ami decimálisan 32, itt \verb|\040| formában szerel. Az escape
szekvenciákat is ismeri, amilyen a \verb|\t| kifejezéssel reprezentált tabulátor
karakter, valamint az újsor karakter (\verb|\n|).

\begin{VerbExample}
tr -d ' '            # szóköz törlése
tr -d '[a-z]'        # angol kisbetűk törlése
tr '[a-f]' '[A-F]'   # nagybetűsítés (hexadecimális számokban)
tr '\t' ' '          # tab karakterek szóközre cserélése
\end{VerbExample}

A \texttt{sed} nem csupán egyszerű helyettesítésre alkalmas, hanem kisebb
programokat (szkripteket) is lehet rá írni, ez utóbbival pillanatnyilag nem
foglalkozunk. A \texttt{tr}-hez hasonlóan alkalmas szöveghelyettesítésre is,
például \texttt{y/abc/def/} formában. Az \texttt{y} jelzi, hogy karaktereket
kell cserélnie, ezután -- itt legalábbis -- perjellel elválasztva következik,
hogy mit kell cserélni, végül az, hogy mire. A program nem fogadja el az
utasítást, ha nem azonos számú karakter szerepel mindkét részben, és még egy
korlát van: nem képes tartományokat kezelni.

A másik lehetőség a szöveghelyettesítés, ekkor az \texttt{s/mit/mire/opciók} az
alak, erről részletesebben \aref{sec:filters-sed}.\ fejezetben lesz szó.



\section{Sorok keresése -- a \texttt{grep} programcsalád}
\label{sec:filters-grep}

A leggyakoribb művelet egy vagy több állományban vagy akár a teljes
könyvtárszerkezetben történő keresés, erre egy egész programcsalád van: a
\texttt{grep} különböző elnevezésekkel -- melyek valójában különböző opciókat
jelölnek. Az első, kötelező paraméter az a szöveg, amelyet az egyes sorokban
keresünk, amely valójában egy minta, reguláris kifejezés, és a \texttt{grep} azt
vizsgálja meg, hogy valamely sor megfelelő része illeszkedik-e erre a
mintára. Működése:\\
\ut{grep minta fájlnevek}

A család általánosan használt tagjai a \texttt{grep}, \texttt{fgrep}, amely a
\texttt{grep -F} opciójának felel meg, valamint az \texttt{egrep}, amely a
\texttt{grep -E} megfelelője. Egyes linuxokon előfordul még az \texttt{rgrep}
is, amely egy shell szkript és a \texttt{grep -R} vagy \texttt{grep -r}
megfelelője. Látható, hogy a jelölés nem egységes, az egyes programok más-más
egy betűs rövidítést használnak ugyanarra az opcióra (a rekurzív hívást következetlenül
jelölik kis- és nagybetűvel is, és sajnos nem ez az egyetlen ilyen opció).

A ``minta'' egy speciális karaktersorozhat, reguláris kifejezés (részletesebben
lásd \aref{cha:regex}.\ fejezetben), azonban nem minden esetben, hiszen a
\texttt{-F} opció vagy az ezzel ekvivalens \texttt{fgrep} használata esetén
egyszerű szövegnek tekinti a \texttt{grep}.

Az előbb felsorolt változatok jelentése: \texttt{grep} önmagában csak alap
reguláris kifejezéseket ismer, \texttt{egrep} pedig kiterjesztetteket is, az
\texttt{fgrep} nem foglalkozik reguláris kifejezésekkel, vagyis ténylegesen a
felsorolt karaktereket tartalmazó sorokat keresi. Az \texttt{rgrep} pedig -- már
ahol létezik -- könyvtárakat elfogad paraméterként, ekkor végignézi az azok
alatt megtalálható összes könyvtárbejegyzést.

Előfordulhat, hogy az illesztendő minta '-' karakterrel kezdődik, ezt viszont a
\texttt{grep} opciónak venné, ezért a \texttt{-e} után kell írnunk a mintát,
hogy helyesen értelmezze a program.

A \verb|^| (kalap) és a \verb|$| (dollár) a reguláris kifejezésekben a sor
elejére illetve végére illeszkedik. A következő példa:\\
\utbeh\verb|grep From  $MAIL|\\
azon sorokat írja ki a levelekből, amelyekben a From karaktersorozat
szerepel, akár legvégén is. Ezzel szemben a\\
\utbeh\verb|grep ^From  $MAIL|\\
csak a From karakterekkel kezdődő sorokat írja ki. A\\
\utbeh\verb,grep /bash$ /etc/passwd | \verb|cut -f1 -d:,\\
azokat a sorokat hagyja meg a \texttt{/etc/passwd} állományból, amelyekben a
bejelentkezési parancsértelmező (login shell) a bash, ez áll ugyanis a sorok
végén. A \texttt{grep} kimenetét a \texttt{cut} kapja meg, 


A megtalált soroknál kiiratható a fájlnév is, valamint a sor sorszáma
is. Alapesetben ha csak egyetlen paraméter esetén
a fájlnév nem jelenik meg, ám több állomány megadásakor már mindenképpen kiírja
az aktuális fájl nevét is. Ez kikényszeríthető a \texttt{-H} opcióval és
letiltható a \texttt{-h} segítségével. A \texttt{-n} hatására jelenik meg az
is, hogy hanyadik sorról van szó. A \texttt{grep -nH} kimenetének szerkezete az
alábbi:\\
\utbeh\texttt{fájlnév:sorszám:eredeti sor}.

\begin{VerbExample}
~>  grep -nH ssh /etc/services
/etc/services:77:ssh              22/udp     # SSH Remote Login Protocol
/etc/services:78:ssh              22/tcp     # SSH Remote Login Protocol
/etc/services:1571:sshell       614/udp     # SSLshell
/etc/services:1572:sshell               614/tcp     # SSLshell
\end{VerbExample}

A forráskódban történő keresésnél zavaró lehet, ha csak a megtalált sorokat írja
ki, a szövegkörnyezet nélkül nehéz megállapítani, hogy éppen melyik is a
számunkra érdekes sor. Erre többféle kapcsoló is van, a legegyszerűbbel a
megtalált sor előtt és után kiirandó sorok számát lehet megadni:
\texttt{-C szám} vagy hosszabb formában: \texttt{--context szám}, ám akár külön
is szabályozható a korábbi és későbbi megjelenített sorok száma a \texttt{-B}
vagy \texttt{--before-context} valamint a \texttt{-A} vagy
\texttt{--after-context} opciók után megadott értékkel.

A teljesség igénye nélkül következzenek a példák.

\begin{VerbExample}
fgrep -h 'int main('  *.c     # Van-e már main() fv. a forráskódban?
                              # Fájlnév nem érdekes.
fgrep -nHi almafa *.txt       # fájlnév:sorszám:megtalált sor formájú
                              # kiírás. Kis- és nagybetűk azonosak.
grep -C3 ^szoveg    # stdin-ről 'szoveg' kezdetű sorok + környezetük
egrep 'a|b' *.txt   # 'a'-t vagy 'b'-t tartalmazó sorok (csak egrep)
\end{VerbExample}

A minta is szerepelhet egy fájlban, ekkor a \texttt{-f} opció után kell megadni.

\begin{VerbExample}
~>  cat fminta
ab?c?d?t
~> egrep -f fminta /etc/fstab | head -n2
root:*:0:0:System Administrator:/var/root:/bin/sh
sshd:*:75:75:sshd Privilege separation:/var/empty:/usr/bin/false
\end{VerbExample}

A példában azokat a sorokat írja ki az \texttt{egrep}, amelyekben van egy `a'
betű, majd esetleg egy `b', amelyet a `c' illetve a `d' követhet, ebben a sorrendben,
de nem biztos, hogy vannak, és utána egy `t' következik. Tehát helyes például a
következő néhány variáció: `at', `abt', `adt', `abct', `acdt'.

\section{A \texttt{sed} (stream editor)}
\label{sec:filters-sed}

A \texttt{sed} egy jól scriptezhető szűrő, az egyes sorokra különbözőképpen
működhet.

A következő példákban a \texttt{sed} standard inputról olvas (vagy a bash
átirányítása révén fájlból, esetleg másik program kimenetéről). Kezdetben
cseréljük le az ``alma'' szót ``körte'' szóra:\\
\ut{sed 's/alma/körte/'}\\
Ez azonban csak az elsőt helyettesíti, mert nem globális, a teljes sorra
vonatkozó a csere. Ennek javítása:\\
\ut{sed 's/alma/körte/g'}\\
Ha a `g' opción kívül az `i'-t is megadtuk volna, akkor a kis- és nagybetűk
között nem tenne különbséget, így azonban például a csupa nagybetűs
``\textsc{alma}'' szót nem cseréli le.

Az utasításokat be is tehetjük egy fájlba, amelynek a nevét a \texttt{-f} opció
után kell megadnunk:

\begin{VerbExample}
sed -f műveletek.sed
\end{VerbExample}

\noindent Sőt, akár szkriptfájlt is írhatunk, például:

\begin{VerbExample}
#!/usr/bin/sed -f
s/alma/körte/gi
\end{VerbExample}

\noindent Ha csak azokban a sorokban szeretnénk cserélni, amelyekben van ``dió''
szó is, akkor a következőképpen néz ki a szkript második sora:

\begin{VerbExample}
/dio/ s/alma/korte/g
\end{VerbExample}

\noindent ez azonban csak az egyik lehetséges megoldás. A másik összetetteb
programoknál hasznos, viszont ilyen egyszerű példánál látszik a legjobban,
hogyan is kell használni. Ezek a címkék, amelyekhez egy vagy több művelet
tartozik, és a szkript bármely részén állhat egy olyan utasítás, hogy a vezérlés
ugorjon ide.

Először tekintsük a szabványos bemenetre érkező adatot:

\begin{VerbExample}
alma alma cseresznye
dio mogyoro alma
dio alma
mogyoro alma meggy narancs
\end{VerbExample}

\noindent Első próbálkozás:

\begin{VerbExample}
#!/usr/bin/env sed -f
/dio/ b csere
:csere
{ s/alma/korte/g }
\end{VerbExample}

A \texttt{b} az ugrás jele, utána egy címke neve van. Egy címke pedig
kettősponttal kezdődik, amely után a neve van. Kapcsoszárójelek között megadható
egy vagy több utasítás, amely akkor hajtódik végre, ha a címkére kerül a
vezérlés. Az eredmény azonban nem a várt lesz:

\begin{VerbExample}
korte korte cseresznye
dio mogyoro korte
dio korte
mogyoro korte meggy narancs
\end{VerbExample}

A gondot az okozza, hogy azok a sorok esetén, amelyekben nem szerepel a ``dio''
szó,  nincsen külön definiált művelet, így a legelső arra alkalmasat hajtja
végre, amely jelen esetben a \texttt{csere} címkéhez tartozó művelet lesz. Kis
keresgélés után a \texttt{d} parancs tűnik megfelelőnek, hiszen törli a
beolvasott sort, ám sajnos nem írja ki őket. Az egyes parancsok külön sorban
kell, hogy szerepeljenek.

\begin{VerbExample}
#!/usr/bin/env sed -f
/dio/ b csere
d
:csere
{ s/alma/korte/g }
\end{VerbExample}

\noindent Az eredmény:

\begin{VerbExample}
dio mogyoro korte
dio korte
\end{VerbExample}

\noindent A \texttt{d}-t \texttt{p} parancsra cserélve megint nem jó a kimenet,
mert bár megjelenik az összes sor, de utána megint ráfut a címkére:

\begin{VerbExample}
alma alma cseresznye
korte korte cseresznye
dio mogyoro korte
dio korte
mogyoro alma meggy narancs
mogyoro korte meggy narancs
\end{VerbExample}

\noindent Ezek szerint a helyes a \texttt{p} majd a \texttt{d}. Nézzük meg:

\begin{VerbExample}
alma alma cseresznye
dio mogyoro korte
dio korte
mogyoro alma meggy narancs
\end{VerbExample}

\noindent Mi történik akkor, ha a címke utasításai \emph{után} még egy
\texttt{p} is szerepel? A csere után ki fogja írni mégegyszer a sort?

\begin{VerbExample}
#!/usr/bin/env sed -f
/dio/ b csere
p
d
:csere
{ s/alma/korte/g }
p
\end{VerbExample}

\noindent Nézzük csak:

\begin{VerbExample}
alma alma cseresznye
dio mogyoro korte
dio mogyoro korte
dio korte
dio korte
mogyoro alma meggy narancs
\end{VerbExample}

\noindent Tényleg ez történt, ugyanis a címkénél egyetlen helyettesítés van
csupán, és bár az adott sorral több teendő nincs, mégsincsen törölve, azaz
kellene még egy \texttt{d} is. Mivel eredetileg ez volt az utolsó utasítás, erre
nem volt szükség.



\section{További szűrők}
\label{sec:filters-others}

Az előzőkben a gyakrabban használt programokat láthattuk, a továbbiakban a
valamivel ritkábban szereplő, de néha hasznos eszközöket vesszük sorra.



\subsection{Teljes fájlok kiiratása}

A \texttt{cat} parancson felül (lásd \aref{ssec:filters-cat}.\ alfejezet) még
néhány hasznos eszköt van, a legegyszerűbb a \texttt{tac}, amely a sorokat
fordított sorrenben írja ki, először az utolsót, és végül a legelsőt. Az
\texttt{nl} a sorokat számozottan írja ki, végül az \texttt{od} decimális,
hexadecimális és oktális értékeit is ki tudja írni a karaktereknek. Ezen utóbbi
program hibakeresésnél is hasznos lehet, mert esetleg a terminál nem jelenít meg
néhány begépelt karaktert, amikor melléütünk, a parancsfájlunk viszont emiatt
nem fut le.

\begin{VerbExample}
tac               # standard input sorait fordítja meg
nl a.txt          # a.txt nem üres sorait számozza
nl -b a a.txt     # minden sor számozottan jelenik meg
od a.txt          # oktális kiírás (2 bájtonként)
od -c a.txt       # karakterenként, egyes karaktereket kódjukkal
od -x a.txt       # hexadecimálisan, 2 bájtonkánt
od -c -t x1 a.txt # karakterek + minden bájt hexadecimális kódja
\end{VerbExample}


% Local Variables:
% fill-column: 80
% mode: latex
% End:
