\chapter{A Linux}%
\label{cha:linux}

A Linux egy Unix-szerű operációs rendszer (OS), vagyis sok mindenben hasonlít
mind a használat, mind a programozás tekintetében. E fejezet az eredetét és az
alapvető működését írja le.


\section{A Linux eredete}

A Linuxot alapvetően nagy tudású hackerek készítik, azonban bárki hozzájárulhat
jobbá tételének. A (mai) hackereknek saját kultúrájuk van, amelynek története az
1960-as évekig nyúlnak vissza ekkor kapott ugyanis az \textsc{mit}
(Massachusetts Institute of Technology) vasútmodellező köre egy
\textsc{pdp-1}-es gépet, \cite{bazar,bazar-web}, amelyhez programozási
eszközöket, nyelvet, sőt, teljes kultúrát alakíttak ki. Ebből jött létre az
egyetem mesterséges intelligencia laborja, amely a 80-as évek vezető
MI-központja volt.

A következő jelentős mérföldkő az ARPAnet kialakításának éve, 1969 volt. E
hálózat eredetileg katonai célokra szolgált, tesztkörnyezetként (amelynek háború
esetén is működnie kellett), mely egész földrészeken átívelt, idővel azonban
rácsatlakoztak az egyetemek is. Az információ most már messzebbre eljuthatott,
ezáltal az \textsc{mit}-beli kutatóközpont hatása is megnőtt.

A \textsc{pdp} sorozat 10-es számú gépe '67-től 15 éven keresztül meghatározó
volt, és bár készült hozzá operációs rendszer is, az MIT hackerei saját
változhatot készítettek, amely az \textsc{its} (Incompatible Time-Sharing System,
Inkompatibilis Időosztásos Rendszer) nevet kapta, számos újítást
tartalmazott. És valóban jól működött. Alapvetően assemblyben íródott, tehát a
kód csak ezen a gépen volt futtatható. Nemegy program \texttt{lisp}-ben íródott,
a legismertebb a ma is használatos \textsc{emacs} programszerkesztő.


Szintén '69-hez kötődik egy másik esemény is. Ken Thompson, a Bell-Laps egyik
hackere megalkotta a Unixot. Eredetileg a Multics nevű OS fejlesztésén
dolgozott, amelyen olyan újításokkal kísérletezett, amelyek elfednék az
operációs rendszer bonyolultságát a felhasználók elől. A rendszer azonban
kezdett túl bonyolulttá válni, így a Bell Labs elállt a piacra dobásától. Ennek
nyomán Thompson nekiállt egy újabb rendszer írni az eddigi ötletei
alapján. Ebben nyújtott segítésget Dennis Rithcie, aki egy új nyelvet dolgozott
ki a kezdeti Unix rendszr számára, a C-t.

Míg eddig minden OS assembly-ben íródott, a Unix már nem, ugyanis már nem volt
fontos a kód adott hardverre történő optimalizálása, s az akkori
fordítóprogramok mellett egy teljes operációs rendszer is megírhatóvá vált egy
magasabb szintű nyelven, például C-ben is.

A C gépfüggetlen volt, így 1978-ra számos géptípusra sikerült átültetni a
Unixot. A nyelv is egyszerű, könnyen fejben tartható volt sok egyéb nyelvtől
eltérően.

A Unix még egy előnnyel rendselkezett: beépítetten tartalmazta az
\texttt{uucp}-t, amely kis sebesságű, megbízhatatlan kapcsolatot tett lehetővé
telefonvonal felhasználásával.

A '80-as évek elején Richard R. Stallman -- akit bejelentkezési azonosítójáról
`rms'-ként is ismerhetünk -- megalapította a Free Software Foundation-t
(\textsc{fsf}-t). A Unix már grafikus felületet is adott, az X Window System-et,
amelyet gyakran csak X-nek nevezünk. A '90-es évek ben az olcsó, nagy
teljesítményű számítógépek terjedtek el.

Bár a hardver fejlődött, nem volt hozzá olcsó szoftver, ezért az
\textsc{fsf}-nél egy Unix rendszermag (\emph{kernel}), a \textsc{hurd}
fejlesztésébe kezdtek, amely azonban nem folytatódott, egészen 1996-ig állt.

Az \textsc{fsf} azonban adott mást: tömérdek ingyenes, nyílt forráskódú
programot, amelyeket \textsc{gnu} programoknak neveztek el. A \textsc{gnu} egy
betűszó, a `\textsc{GNU} is Not Unix' rövidítése. A be nem fejezett
rendszermaggal együtt a \textsc{gnu/hurd} elnevezést kapta.

Ezekkel az eszközökkel állt neki egy finn egyetemista -- Linus Torvalds -- egy
rendszermag elkészítésének. 1991 nyarára sikerült egy működő
lemezmeghajtó-vezérlő kódot írnia. E kernel a Linux nevet kapta, és a
felhasználói programokkal együtt gyakran  GNU/Linux-nak nevezik, bár ez ma már
nem pontos, hiszen számtalan nem \texttt{gnu} program is van rajta.

Ugyancsak ezekben az időkben portolták az egyik Unix változatot, a
\textsc{bsd}-t -- melyet a Berkeley egyetemen készítettek, és a név a Berkeley
Software Distribution rövidítése -- 386-os gépekre.

A Linuxot igazán sikereressé az Internet és a világháló elterjedése tette.

\section{A fájlrendszer}
A Linux fájlrendszer egyetlen, összefüggő fából áll, a merevlemez különböző
partícióit ennek a fának valamely csúcsára lehet rácsatolni, feltéve ha az adott
csúcs egy könyvtár. Néhány esetben még az is elvárás, hogy a könyvtár üres
legyen. 

A File System Hierarcy Standard\acite{fhs} (\textsc{fhs}) tartalmazza, hogy mely
könyvtáraknak mi a szerepük, ezek közül sok opcionálos, vagy csupán a régebbi
rendszerekkel való kompatibilitást hivatott biztosítani. néhány példát
\aref{tab:linux-fs}.\ táblázatban láthatunk.


\begin{table}%
\center%
\begin{tabular}{@{}ll@{}}\toprule
Könyvtár & Funkció\\\midrule
/boot & A Linux kernelét tartalmazza, induláskor használt\\
/bin  & A rendszer indulásakor is használt programok\\
/dev  & Eszközfájlok a hardverek kezeléséhez\\
/root & A root felhasználó home könyvtára\\
/sbin & Rendszerinduláskor is használt programok a root számára\\
/usr  & A felhasználói programok és azok adatai\\
/usr/bin & Mindenki számára elérhető programok, melyek\\
& nem szükségesek a rendszer elindulásakor
\\\bottomrule
\end{tabular}
  \caption{Néhány könyvtár és azok funkciója}
  \label{tab:linux-fs}
\end{table}

%\section{A Linux rendszer elindulása}


%\section{A legfontosabb beállítások}


%\section{Használat}

% Local Variables:
% fill-column: 80
% mode: latex
% End:
