\chapter{Vezérlési szerkezetek}%
\label{cha:vez-szerk}

Egy shell szkript a programokhoz hasonlóan vezérlési szerkezetekből áll,
melyeknek három alapvető változata van, a szekvencia, a ciklus és az
elágazás, ezek mindegyike egy-egy összetett utasítás.

Egyes utasításokat összefogva külön is definiálhatunk, melyek így egy
függvényt alkotnak, és a szkript bármely pontján meghívhatjuk e függvényt,
mintha annak utasításai az adott helyen lennének (vagyis úgy viselkedik, mint
bármely program).

Először tekintsük a szekvenciát, amely utasítások egymásutánját jelenti. Amint
az egyik utasítás (program) végetért, elindulhat a következő, ám egyszerre
mindig csak egyetlen egy fut. Az előbb futó program hatása érvényesül a később
futónál, például létrehoz egy fájlt, vagy megváltoztat egy változót, akkor
mindkét változtatás ``látszik'' a későbbinél\footnote{A változó módosítása csak
akkor látszik, ha az első utasítás a bash része volt, és nem állt egy
alshellben.}.

A szekvenciát alkotó utasításokat a pontosvessző (';') vagy pedig az újsor
karakter választja el. Amennyiben minden utasítás külön sorban van, akkor nem
szükséges kitenni a pontosvesszőt az utasítások végére, bár ki lehet.

Az utasítások után állhat másik karakter is, az ``és'' (ampersand, '\&'), amely
az előtte álló utasítást a háttérbe küldi, vagyis a parancsértelmezővel
\emph{párhuzamosan} fut, ez azonban már nem szekvencia!

Egyetlen, összetett  utasításnak tekinthető a csővezeték is, ahol a cső ('|')
karakterrel elválasztva szerepelnek a programok -- valójában szintén utasítások
--, amelyről \aref{cha:szűrők}.\ fejezetben volt szó.

Az elágazás és a ciklus annyiban hasonló, hogy mindkettő esetén meg kell adnunk
egy vagy több feltételt, amelynek (nem) teljesülése esetén futnak le bizonyos
utasítások. Egy feltétel mindig valamilyen utasítás (program), amelynek státusza
dönti el az igazságértéket. Ha a státusz nulla, akkor a feltétel igaz, ugyanis a
program sikeresen lefutott. A nem nulla érték valamilyen hibát jelez, ezért ez a
hamis állapotot jelzi.

\section{Feltételek}

A feltétel ellenőrzése mindig valamely utasítás státuszának 0-val egyezőségét
jelenti. Ez lehet egy program, de lehet ``hagyományos értelemben vett''
összehasonlítás is, például  $i \leqq 10$ vagy éppen \texttt{A =
  "almafa"}. Valójában ezen utóbbiakat is egy program használatával
vizsgálhatjuk meg, az újabb bash változatok viszont már ezt beépített
parancsként tartalmazzák (azonos név, azonos funkciók és használat mellett).

Kétféle elnevezés van rá, az egyik a \texttt{test} (eredetileg
\texttt{/bin/test}), a másik pedig a szögletes zárójel, `\texttt{[}'. Ezen
utóbbi egy symlink a \texttt{test}-re a \texttt{/bin} könyvtárban, valamint
azonos jelentésű beépített parancs is.

Amikor a `\texttt{[}' változatot használjuk, akkor meg kell adnunk a csukó
zárójelet ('\texttt{]}') is a végén. Tehát a következő két forma teljesen azonos
jelentést hordoz:

\begin{VerbExample}
test kifejezés
[ kifejezés ]
\end{VerbExample}

Az előző példa alapján az $i \le 10$ összehasonlítás a \texttt{test} paranccsal
két alakjával a következő:

\begin{VerbExample}
test  $i -lt 10
[  $i -lt 10 ]
\end{VerbExample}

A kifejezés lehet összehasonlítás, valamint a fájlrendszer valamely részének
tulajdonságát ellenőrző, illetve sztringek hoszzát vizsgáló kifejezés
is. Először tekintsük meg az összehasonlításokat, melyeknek szintén két
változata van, az egyik matematikai összehasonlítást jelent, a másik pedig
sztringekre vonatkozót.

A matematikai alakokat (ahol \texttt{A} is és \texttt{B} is lehet számkonstans,
vagy éppen egy változóbehelyettesítés is, például \verb|$i|)
\aref{table:vez-feltetelek}.\ táblázat első része tartalmazza.
 
\begin{table}[tbh]\center
\center\begin{tabular}{ll}\toprule
Kód & Jelentés\\\midrule
\texttt{A -lt B} & \texttt{A $<$  B}\\
\texttt{A -le B} & \texttt{A $\leq$  B}\\
\texttt{A -eq B} & \texttt{A $=$  B}\\
\texttt{A -ne B} & \texttt{A $\neq$  B}\\
\texttt{A -ge B} & \texttt{A $\geq$  B}\\
\texttt{A -gt B} & \texttt{A $>$  B}\\\midrule
\texttt{S1 < S2} & Igaz, ha \texttt{S1} lexikografikusan kisebb, mint
\texttt{S2}\\
\texttt{S1 > S2} & Igaz, ha \texttt{S1} lexikografikusan nagyobb, mint
\texttt{S2}\\
\texttt{S1 != S2} & Igaz, ha a két string nem egyenlő\\
\texttt{S1 == S2} & Igaz, ha azonosak\\
\texttt{S1 = S2} & Igaz, ha azonosak (\textsc{posix} kompatibilis jelölésmód)\\
\texttt{-z str} & Igaz, ha \texttt{str} hossza 0\\
\texttt{str} & vagy:\\
\texttt{-n str} & Igaz, ha \texttt{str} nem nulla\\\midrule
\texttt{-e file} & Igaz, ha a \texttt{file} létezik\\
\texttt{-f file} & Igaz, ha a \texttt{file} reguláris fájl\\
\texttt{-d file} & Igaz, ha a \texttt{file} könyvtár\\
\texttt{-L file} & Igaz, ha a \texttt{file} symlink\\
\texttt{-b file} & Igaz, ha a \texttt{file} blokkos eszköz\\
\texttt{-c file} & Igaz, ha a \texttt{file} karakteres eszköz \\
\texttt{-r file} & Igaz, ha a \texttt{file} olvasható\\
\texttt{-w file} & Igaz, ha a \texttt{file} írható\\
\texttt{-x file} & Igaz, ha a \texttt{file} futtatható\\
\texttt{file1 -nt file2} & Igaz, ha a \texttt{file1} újabb,
 mint \texttt{file2}\\
\texttt{file1 -ot file2} & Igaz, ha a \texttt{file1} régebbi,
 mint \texttt{file2}\\

\bottomrule
\end{tabular}
 \caption{Feltételek}
  \label{table:vez-feltetelek}
\end{table}


A sztringek esetén az összehasonlítás lexikografikus rendezés szerint történik,
amelyik előrébb szerepel, az a ``kisebb''. Ez a rendezés függ a nyelvi
beállításoktól. Sőt, sztringek esetén a hossz nulla vagy nem nulla voltát is
lehet ellenőrizni. A lehetőségeket \aref{table:vez-feltetelek}.\ táblázat
második részében láthatjuk.

Az utolsó részben a fájlokra és könyvtárakra vonatkozó legtöbb vizsgálat
szerepel, például megvizsgálhatjuk, hogy egy adott könyvtárbejegyzés (reguláris)
fájl-e, vagy olvasható-e, vagy éppen azt, hogy két fájl közül melyik az újabb:

\begin{VerbExample}
[ $F1 -nt $F2]
\end{VerbExample}

\noindent státusza 0, ha az \texttt{\$F1} az újabb.


\section{Elágazás}

Az elágazás egy vagy több feltételből áll, valamint az egyes feltételekhez
tartozó utasításlistákból. Ha valamelyik feltétel teljesül, akkor a hozzá
tartozó listában lévő utasítások fognak lefutni. Az első feltételt kötelező
megadni, valamint a hozzá tartozó utasításokat is. Amennyiben nem teljesül a
feltétel, akkor egy másik utasítássorozat is szerepelhet, amely ekkor fut le.
A legegyszerűbb változat tehát a következő (egy feltétel + utasítások):

\begin{VerbExample}
if feltétel
then
    utasítások
fi   
\end{VerbExample}

\noindent Amennyiben egyetlen sorba szeretnénk írni, akkor nem újsor karakter,
hanem pontosvessző kerül az egyes részek közé:

\begin{VerbExample}
if feltétel; then utasítások; fi
\end{VerbExample}

\noindent Az elágazás mindig az \texttt{if} kulcsszóval kezdődik, és a
\texttt{fi} szóval végződik, és minimálisan a \texttt{then} utáni utasítások
szükségesek, amelyek a feltétel teljesülése esetén futnak le. Gyakran ezt az
ágat ``then-ágnak'' nevezzük.

Amennyiben olyan utasításokat is meg szeretnénk adni, amelyek a feltétel vagy
feltételek nem teljesülése esetén futnak le (amikor mindegyik hamis), akkor
azokat az \texttt{else} kulcsszó után írhatjuk (``else-ág''):

\begin{VerbExample}
if feltétel
then
   utasítások      # ha a feltétel igaz
else
   utasítások      # ha a feltétel hamis
fi
\end{VerbExample}

\noindent Gyakran további feltételeket is meg szeretnénk adni, ezért ezt a bash
támogatja, nem szükséges a az \texttt{else} után újabb elágazást megadnunk,
hanem az \texttt{elif} után további feltételeket adhatnunk meg. Ez az új
feltétel implicit módon tartalmazza az összes korábbi tagadását, hiszen
ellenkező esetben valamely korábbi ág futna le, nem pedig az ehhez az
\texttt{elif}-hez tartozó ág. Egy \texttt{if} utasításban bárhány \texttt{elif}
utáni feltétel lehet, de csak egyetlen \texttt{else} utáni ág, hiszen ez utóbbi
fut le ``minden egyéb esetben'', amikor a feltételek nem teljesülnek.

\begin{VerbExample}
if feltétel1; then
   utasítások1;
elif feltétel2; then
   utasítások2
...
elif feltételn; then
   utasításokn
else
   utasítások
fi
\end{VerbExample}

\noindent Tekintsünk egy egyszerű példát:

\begin{VerbExample}
if [[ $# -eq 0 ]]; then
  echo Hiányzó paraméter! >&1
  ecit 1
elif [[ "X$1" = "X--help" ]]; then
  echo Segítség!
  exit 0
elif [[ -d "$1" ]]; then
  echo Ez könyvtár!
  exit 2
elif [[ -f "X$1" && -r "$1" ]]; then
  cat $1
fi 
\end{VerbExample}

\noindent A második ellenőrzés esetén szükséges az ``X'' beszúrása, ugyanis
előfordulhat, hogy az első paraméter véletlenül épp a ``-f'' vagy a ``-d'',
amelyek mindegyike feltételezi, hogy utána az összehasonlításban egyetlen dolog
áll. A bash az ellenőrzést csak a változók behelyettesítése után végzi el,
amikor már nem tud különbséget tenni, hogy az a bizonyos ``-f'' konkrétan honnan
is származott: paraméterből, vagy a szkriptben eredetileg is benne volt.

\subsection{A \texttt{case} utasítás}

Az elágazás egy speciális esete a \texttt{case}, amikor valamilyen értéket
hasonlítunk össze egy vagy több másikkal, és az első találathoz tartozó ág fut
le. A lehetséges értékhalmaz programozási nyelv-függő, a shell esetén
sztringeket jelent.

Minden ághoz legalább egy lehetséges érték tartozik, és az utasításokat dupla
pontosvessző (';;') zárja. A teljes elágazás végét a kezdő kulcsszó fordított
irányú megadása (\texttt{esac}) jelzi.

Tekintsünk rögtön egy példát, amelynél a paraméterek száma (sztringként
kezelve!) a feltétel:

\begin{VerbExample}
case $# in
0) echo egyetlen paraméter sincs;;
1|2) echo kevesebb, mint 3 paraméter van;;
3*) echo három, vagy harmincvalahány.... paraméter van;;
6);; # nem csinál semmit itt
*) echo minden más eset. Legalább 4 paraméter van, az első: "'$1'";;
\end{VerbExample}

A fenti példa az összes lehetséges alakot lefedi, a csillag bármilyen karakter
nulla vagy több előfordulását jelzi, ahogy reguláris kifejezésekben a '.*'. A
cső karakter ('|') az egyes eseteket választja el egymástól, ezáltal több ág is
összevonható, egyszerre kezelhető.

Az utolsó ág nem kötelező, azonban nagyon gyakran a csillag a feltétel, vagyis
bármely egyéb, korábban fel nem sorolt esetben az itt megadott utasítások futnak
le.

A példában az első ág a '0' esetén fut le, a második az '1' és a '2' esetén, a
harmadik pedig a '3' és a legalább kétjegyű, hárommal kezdődő számok esetén
hajtódik végre. Bizonyos esetekben szükséges olyan ág is, amely nem csinál semmi
mást, mint nem engedi ráfutni a vezérlést valamely későbbi ágra, ilyen most a
negyedik, 6 paraméter esetén lefutó ág. Ezután van a minden fel nem sorolt
esetet lefedő ág, amelyet a '*' jelez.

\subsection{Egyéb változatok}

Az \texttt{if} utasítás kiváltható egyszerűbb alakkal is, tipikusan akkor,
amikor  csak a ``then-ág'' vagy csak az ``else-ág'' szükséges. Természetesen
megoldható az is, hogy mindkettő szerepeljen, sőt, ezzel a változattal jóval
bonyolultabb kifejezések is kényelmesen leírhatóak.

A feltétel mindig valamely program státusza, melynek függvényében az utána lévő
egyéb programok lefutnak. Ha a sikeres lefutás esetén kell egy másiknak is
lefutnia, akkor köztük a  ``dupla és'' (\texttt{\&\&}) szerepel, ha pedig
sikertelen esetben, akkor a  ``dupla vagy'' (\texttt{||}), végül státusztól
függetlenül le kell futnia, akkor pontosvessző (azaz szekvenciáról van szó).

A következő példában szerepelni fog a \texttt{true} parancs, melynek státusza
mindig 0, azaz sikeresen lefut, továbbá a \texttt{false}, melyé nem nulla,
vagyis mindig sikertelenül ér véget.

\begin{VerbExample}
false && true && echo a || echo b && echo c
\end{VerbExample}

\noindent A kimenete mindig a következő lesz:\\
\ut{b}\\
\ut{c}

Ennek oka az, hogy a legelső utasítás státusza nem nulla, ezért a tőle
\texttt{\&\&} karakterekkel elválasztott parancsok sem futhatnak, csak a
\texttt{||} utáni rész, amelyből az első a ``b'' karaktert írja ki. Ez utóbbi
nem jelez hibát, ezért az utána lévő utasítás is lefuthat, amely a
``c'' karaktert írja ki.

Tehát összefoglalva a kétféle jelölés:

\begin{VerbExampleNum}
utasítás1 && utasítás2
utasítás3 || utasítás4
\end{VerbExampleNum}

\noindent Az első esetben amennyiben az \texttt{utasítás1} hiba nélkül végetért,
lefut az \texttt{utasítás2} is, különben pedig nem. A (2) esetben pedig pont
fortdítva, akkor fut le az \texttt{utasítás4}, hogyha az \texttt{utasítás3}
hibát jelzett, azaz a státusza nem nulla.


\section{Ciklusok}

A legtöbb programozási nyelv -- így a szkriptnyelvek közé tartozó bash szkriptek
is -- támogatják a ciklus használatát, amelyet általában \texttt{while}
ciklusnak neveznek, valamint egyéb változatokat, amelyek a \texttt{while} és a
további vezérlési szerkezetekkel is megadhatóak.

A \texttt{while} ciklus egy feltételből és egy ciklusmagból áll, ez utóbbi
pontosan egy -- esetleg összetett -- utasítást jelent. A ciklus mindaddig
ismétli ezt az utasítást, amíg a feltétel teljesül, amely parancsfájlok esetén
egy utasítás sikeres lefutását jelenti, amikor a státusza nulla.

A ciklus státusza a magban legutoljára lefutott utasítás státusza lesz.

Legalább négy sorból áll, ha pedig kevesebből, akkor a sorvége karakter helyett
pontosvessző áll a részei között, akárcsak az elágazás esetén.

\begin{VerbExample}
while program
do
   utasítások
done

while program; do utasítások; done
\end{VerbExample}

\noindent Például végtelen ciklust ad a következő, amely pontosan ugyanazt
teszi, mint a \texttt{yes} parancs:

\begin{VerbExample}
#!/bin/bash
T=${1:-y}
while :
do
   echo $T
done
\end{VerbExample}

\noindent A ciklus felírható elágazás és ciklus segítéségével, amikor eggyel
kevesebbszer fut le a ciklusmag, amennyiben legalábbis először igaz volt a
feltétel -- a program hibamenentesen végetért. Ha az elágazás feltétele nem
teljesült, akkor az else-ág tartalma az üres utasítás, a \textsc{skip}, amelyet
ki sem írunk:

\begin{VerbExample}
if program ;
then
    utasítások
    while program; do
        utasítások
    done
fi
\end{VerbExample}


\subsection{A \texttt{for} ciklusok}

Bár csupán a \texttt{while} segítségével is megoldható lenne bármely ciklust
igénylő feladat, a gyakran előforduló egyéb alakokra van külön változat, a
\texttt{for} ciklus. Sőt, ez utóbbinak is több variánsa van, az egyik egy lista
elemeit járja végig, a másik pedig a C/C++ nyelveknél megszokott forma.

A lista egymás után szereplő elemei egy, a ciklus definiálásakor megadott
változóba kerülnek, majd pedig minden újabb elem esetén lefut a ciklusmag. Az
alapértelmezett elválasztó karakterek a szóköz, a tabulátor valamint az újsor,
ez azonban módosítható az \texttt{IFS} változó beállításával.

\begin{VerbExample}
# alapértelmezett értékkel:
# IFS=$' \t\n'
for i in első második harmadik
do
  echo $i
done
\end{VerbExample}

\noindent A következő példa egy elsőre bonyolultnak tűnő feladatra ad egyszerű
megoldást, nevezetesen, hogy bizonyos nevű fájl szerepel-e a \texttt{PATH}
környezeti változóban megadott könyvtárakban (sh végződésű fájlok).

\begin{VerbExample}
IFS=:
for dir in $PATH
do
  ls $dir | grep sh$ && echo $dir
done
\end{VerbExample}

A \texttt{for} ciklus másik alakja a \texttt{while} ciklusra hasonlít, ugyanis
van egy feltétele, azonban még két utasítás szerepel benne, amelyek a ciklus
elidulása előtt illetve minden egyes iteráció (ciklusmag lefutás) után
végrehajtódnak. Ez az alak a következő:

\begin{VerbExample}
for ((kif1 ; kif2 ; kif3 ))
do
   utasításlista
done
\end{VerbExample}

\noindent Tehát a \texttt{kif1} kifejezés a ciklus elindulása előtt értékelődik
ki (egy értékadás szokott lenni, például \texttt{i=0}), a \texttt{kif2}
kifejezés a ciklus feltétele, amely minden egyes iteráció előtt kiértékelődik,
 a \texttt{kif3} pedig az \texttt{utasításlista} után értékelődik ki (hajtódikk
 végre). Erre példa az első 10 pozitív egész szám kiírása:

\begin{VerbExample}
for ((i=1; i<=10; ++i))
do
   echo $i
done
\end{VerbExample}

\noindent A szerkezet azért is speciális, mivel a \$ jelet nem szükséges
használni a dupla zárójelen belül, továbbá a kisebb-egyenlő reláció jele is a
\verb|<=|, pedig bármely más környezetben ez az \texttt{=} nevű fájlból történő
olvasás (átirányítás) lenne (illetve a \textless reláció sztringek közt, lásd
\aref{table:vez-feltetelek}.\ táblázatot).

Az előző átírható while ciklussá, kétféleképpen is, a második a hagyományos
eszközökkel történő megadás, ami a régebbi shellekben is működne.

\begin{VerbExample}
((i=1))
while ((i <= 10))
do
  echo $i
  ((++i))
done

i=1
while test $i -lt 10
do
  echo $i
  i=$[ $i + 1]
done
\end{VerbExample}



% Local Variables:
% fill-column: 80
% mode: latex
% End:

